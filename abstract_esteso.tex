%\documentclass[11pt]{memoir}
\documentclass[10pt,a4paper,oneside,reqno]{book}
\usepackage[usenames,dvipsnames]{color}
\usepackage{amsmath}%
\usepackage{amsfonts}%
\usepackage{amssymb}%
\usepackage{amsthm}%
\usepackage{amscd,amsthm}
\usepackage{graphicx}
\usepackage{dsfont}
\usepackage[hidelinks]{hyperref} %hyperlink
\usepackage{subcaption} %NON SO SE SERVE
\usepackage{enumitem} %enumerazioni con le i
\usepackage{booktabs} %tabelle più belle
\usepackage{placeins} %forza le immagini in posizione
\usepackage{mathabx}
\usepackage{mathtools}
\usepackage{epsfig}
\usepackage{caption} %caption fuori da ambienti float
\usepackage{appendix} %appendice per i codici
\usepackage{dsfont}
\usepackage{color}
\usepackage{listings}
\usepackage{xcolor} %per colorare il testo
\def\R{\mathbb{R}}
\def\Z{\mathbb{Z}}
\def\H{\mathbb{H}}
\def\A{\mathbb{A}\mathrm{d}\mathbb{S}}
\def\D{\mathbb{D}}
\def\S{\R\textup{P}^1}
\def\AS{\widetilde{\mathbb{A}\mathrm{d}\mathbb{S}}{}}
\def\1{\mathds{1}}
\def\PSL{\text{PSL}(2,\R)}
\def\T{\R\text{P}^1\times\R\text{P}^1}
\def\CF{\mathcal{C}(\Lambda_\varphi)}
\def\hf{T_{\text{Id}}^{1,+}}
\newcommand\norm[1]{\lVert#1\rVert}
\newcommand\red[1]{\textcolor{red}{#1}}
\newcommand{\I}{I}
\newcommand{\II}{I\hspace{-0.1cm}I}
\newcommand{\III}{I\hspace{-0.1cm}I\hspace{-0.1cm}I}
\newcommand{\JJ}{\mathcal{J}\!}

\theoremstyle{plain}
\newtheorem{theorem}{Theorem}[chapter] % reset theorem numbering for each chapter
\theoremstyle{definition}
\newtheorem{definition}[theorem]{Definition} % definition numbers are dependent on theorem numbers
\newtheorem{lemma}[theorem]{Lemma} % same for example numbers
\theoremstyle{plain}
\newtheorem{proposition}[theorem]{Proposition}
\theoremstyle{plain}
\newtheorem{observation}[theorem]{Remark}
\renewcommand*{\proofname}{Proof}
\newcommand\restr[2]{\ensuremath{\left.#1\right|_{#2}}}
\newtheorem{corollary}[theorem]{Corollary}


\newtheoremstyle{mystyleNormalFont}% name
  {0.5 cm}% Space above
  {0.5 cm}% Space below
  {\normalfont}% Body font
  {}% Indent amount
  {\bfseries}% Theorem head font
  {:}%Punctuation after theorem head
  {3mm}%Space after theorem head
  {\thmname{#1}\thmnumber{ #2}\thmnote{ (#3)}}% theorem head spec

  \theoremstyle{mystyleNormalFont}
 \newtheorem{example}[theorem]{Example}

\definecolor{mygreen}{RGB}{28,172,0} % color values Red, Green, Blue
\definecolor{mylilas}{RGB}{170,55,241}
\makeatletter
\makeatother


\usepackage[utf8]{inputenc}
\usepackage[english]{babel}
\usepackage{mathrsfs}
\usepackage{array}
\usepackage{rotating}
\usepackage{multirow}
\usepackage{amsmath,amsthm}
\usepackage{mathtools}
\usepackage{stmaryrd}
\usepackage{tikz-cd}
\usepackage{csquotes}
\usepackage{lscape}
\tikzcdset{
    startcode/.style={
        start anchor={[xshift=0.35cm]south west}
        },
    endcode/.style={
        end anchor={[xshift=0.35cm]north west}
    }
}
\usepackage{tikz}
\usepackage{upgreek}
\usepackage{varwidth}
\usepackage{hyperref}
\usepackage{biblatex}
\usepackage[colorinlistoftodos]{todonotes}
\addbibresource{refs.bib}
\usetikzlibrary{arrows}
\newcommand{\encadrer}[1]{{%
  
  \fbox{%
    \begin{varwidth}{\dimexpr\linewidth-2\fboxsep-2\fboxrule}
      #1%
    \end{varwidth}%
  }
  \par}%
}



%\chapterstyle{veelo}
%\checkandfixthelayout
\setcounter{tocdepth}{2}


\begin{document}
\include{titolo}

Minimal Lagrangian maps have played an important role in the study of hyperbolic
structures on surfaces. As observed independently by Labourie and Schoen
, given two closed hyperbolic surfaces $(\Sigma_1, h_1)$ and $(\Sigma_2, h_2)$, there exists a unique minimal Lagrangian diffeomorphism in the homotopy class of every diffeomorphism $\Sigma_1 \to \Sigma_2$.
Alternative proofs have been provided later, and in this thesis we will concentrate on one proof in the context of Anti-de Sitter threedimensional geometry.
\chapter{Preliminaries on Lorentzian geometry}

Let us recall in this chapter classical results for Lorentzian manifold of constant sectional curvature.
\section{Maximal isometry group and geodesic completeness}
Lorentzian manifolds of constant curvature $K$ are loally isometric, more precisely
\begin{lemma}
    Let $M$ and $N$ be Lorentzian manifolds of constant curvature $K$. Every linear isometry $L: T_x M \to T_y N$ extends to an isometry $f: U \to V$ where $U$ and $V$ are neighborhoods of $x$ and $y$ respectively. Any two extensions $f:U\to V$ and $f' : U' \to V'$ coincide on $U \cap U'$. Moreover, if $M$ is simply connected and $N$ is geodesically complete $L$ extends to a local isometry $f:M\to N$.
\end{lemma}
As a consequence there is a unique simply connected, geodesically complete Lorentzian manifold of constant curvature $K$ up to isometries.\\
Another consequence is that the group of isometries $\text{Isom}(M)$ can be realized as a subset of $\text{ISO}(T_{x_o}, TM)$,and the maximal dimension of $\text{Isom}M$ is $\text{dim}(O(n+1)) + n +1 = (n+1)(n+2)/2$.\\
\begin{definition}
    We say that M has \textit{maximal isometry group} if $\text{Isom}(M) \subset \text{ISO}(T_xM, TM)$ is a bijection. (In particular it follows that the dimension of the isometry group is maximal.)
\end{definition}
From the corollary it follows that every simply connected Lorentzian manifold $M$ has maximal isometry group if it is geodesically complete and has constant curvature . The converse holds:
\begin{lemma}
    If $M$ is a Lorentzian manifold with maximal isometry group then $M$ has constant sectional curvature and is geodesically complete.
\end{lemma}
For simply connected Lorentzian manifold there is the following classification result:
\begin{proposition}
    Let $M_K$ be a simply connected Lorentzian manifold of constant curvature $K$ with maximal isometry group. If $M$ is a Lorentzian manifold of constant curvature $K$ then:
    \begin{itemize}
        \item M is geodesically complete if and only if there is a local isometry $p:M_K \to M$ which is a universal covering.
        \item M has maximal isometry group if and only if $\text{Aut}(p:M_K \to M)$ is normal in $\text{Isom}M_K$
    \end{itemize}
\end{proposition} 
It follows that any isometry between connected open subsets of a Lorentzian manifold $M$ with
maximal isometry group extends to a global isometry. In particular if $M_K$ is a Lorentzian
manifold of constant sectional curvature $K$ with maximal isometry group, than any Lorentzian
manifold $M$ of constant sectional curvature $K$ admits a natural $(\text{Isom}(M_K),M_K)$-structure
whose charts are isometries between open subsets of $M$ and open subsets of $M_K$.
We will refer to Lorentzian manifolds of constant sectional curvature $K$ with maximal isometry group as models of constant sectional curvature $K$. In the following chapters we will study several models of constant sectional curvature $-1$ called models of Anti-de Sitter geometry.

\chapter{Anti-de Sitter Space} \label{chapter:2}

\red{Numerare equazioni, aggiungiere riferimenti}\\
In this chapter we construct models of Anti-de Sitter geometry, pointing out analogies with the models of hyperbolic space.

\section{The quadric model}
We denote by $\R ^{n,2}$ the real vector space $\R ^{n+2}$ equipped with the quadratic form
\[
q_{n,2}(x) = x_1^2 + \cdots + x_n^2 - x_{n+1}^2 - x_{n+2}^2,
\]
by $\langle v,w \rangle_{n,2}$ the associated symmetric form and by $O(n,2)$ the group of linear transformation of $\R^{n+2}$ which preserve $q_{n,2}$.\\
We define
\[
\H^{n,1} := \{ x\in \R^{n,2} \ | \ q_{n,2} = -1 \},
\]
that is a smooth connected submanifold of $\R^{n,2}$ of dimension $n+1$. The tangent space $T_x \H^{n,1}$ coincides with the orthogonal space
\[
x^\perp = \{ y\in\R^{n,2} \ | \ \langle x,y \rangle=0 \}.
\]
The restriction of the symmetric form $\langle \cdot , \cdot \rangle$ to $T\H^{n,1}$ induced a Lorentzian metric on $\H^{n,1}$ with sectional curvature $-1$.\\
We remark that the hyperbolic space $\H^n$ is isometrically embedded in $\H^{n,1}$ as the submanifold defined by $x_{n+2}=0,\ x_{n+1}>0$.\\
$\H^{n,1}$ as a Lorentzian manifold has maximal isometry group $O(n,2)$.

\section{The Klein model}
Let us define
\[
\A^{n,1} := \H^{n,1} / \{ \pm \1 \}.
\]
It is a Lorentzian manifold, has maximal isometry group, and is therefore a model of constant sectional curvature $-1$.
One shown that the center of the isometry group of $\A^{n,1}$ is trivial, making it the \emph{minimal} model of AdS geometry, in the sense that any other model is a covering of $\A^{n,1}$.\\
By definition $\A^{n,1}$ is identified with the subset of $\mathbb{R} \text{P}^{n+1}$ given by the timelike direction of $\R^{n+2}$:
\[
\A^{n,1} = \{ \left[ x \right]\in \mathbb{R} \text{P}^{n+1} \ | \ q_{n,2}<0  \}.
\]
Like in hyperbolic geometry, the boundary of $\A^{n,2}$ is the projectivization of the set of lightlike vectors in $\R^{n,2}$:
\[
\partial\A^{n,1} = \{ \left[ x \right]\in \mathbb{R} \text{P}^{n+1} \ | \ q_{n,2}=0  \}.
\]
Isometries of $\A^{n,1}$ induce projective transformation which preserve $\partial \A^{n,1}$.\\

\section{The Poincaré model for the universal cover}
$\H^{n,1}$ is not simply connected, being homeomorphic to $\R^n \times S^1$.
Let $\H^n$ be the hyperboloid model of hyperbolic space. Then
\[
\pi(y,t) = (y_1 , \cdots , y_n, y_{n+1} \cos(t) , y_{n+2} \sin(t))
\]
is a map $\pi : \H^n \times \R \to \H^{n,1}$ that is a covering with deck transformation of the form $(y,t) \to (y, t+2k\pi)$ for $k \in \Z$. Composing $\pi$ with the double quotient the universal cover for the projective model $\A^{n,1}$ is clearly  $\AS ^{n,1} :=  \H^n \times \R$.\\
Pulling back the metric of $\H^{n,1}$ over $\H^n \times \R$ we get a simply connected model of Anti-de Sitter geometry.\\
Using the Poincaré model of the hyperbolic space (namely $\D^n$), $\AS^{n,1}$ has the model $\D^n \times \R$ equipped with the metric
\[
    \frac{4}{(1-r^2)^2} (dx_1^2 + \cdots + dx_n^2) - \Big( \frac{1+r^2}{1-r^2} \Big)^2 dt^2.
\]
The Poincaré model of the AdS geometry is then the cylinger $\D^n \times \R$ equipped with this metric.
Another conformally equivalent metric on the Poincaré model is given by:
\begin{equation}\label{emispherical}
    \frac{4}{(1+r^2)^2}(dx_1^2+\dots+dx_n^2)-dt^2
\end{equation}
The first term in the equation is the form of the spherical metric on a hemisphere, pulled-back to the unit disk by the stereographic projection.
We will call such a metric hemispherical and will denote it by $g_{\mathbb{S}^n}$. 

\section{Geodesics}
\textit{In the quadric model}. Given $x\in \H^{n,1}$ and $v \in T_x \H^{n,1}$ we want to determine the geodesic through $x$ with speed $v$.
If $v$ is lightlike, then
\[
\gamma(t) = x + tv
\]
is a geodesic of $\R^{n,2}$ contained in $\H^{n,1}$, hence $\gamma$ is a geodesic for $\H^{n,1}$.\\
As in the hyperbolic, timelike and spacelike geodesics are obtained as the intersection of $H^{n,1}$ with a timelike plane. The geodesic is therefore given by the expression
\[
\gamma(t) = \cosh(t)x + \sinh(t)v
\]
if $q_{n,2}(v)= 1$ and 
\[
\gamma(t) = \cos(t)x + \sin(t)v
\] 
if $q_{n,2}(v)= -1$.\\

\textit{In the Klein model}. In analogy with the hyperbolic case, geodesics in the Klein model are intersection of projective lines with the domain $A^{n,1} \subset \mathbb{R}\text{P}^{n+1}$. From the study of geodesics in the quadric model follows that
\begin{itemize}
    \item Timelike geodesics correspond to projective lines that are entirely contained in $\A^{n,1}$, are closed non-trivial loops and have length $\pi$,
    \item Spacelike geodesics correspond to lines that meet $\partial\A^{n,1}$ transversally in two points. They have infinite length, 
    \item Lightlike geodesics correspond to lines tangent to $\partial\A^{n,1}$.
\end{itemize}

\textit{In the universal cover}. In the universal cover $\AS^{n,1}$ geodesics are the lifts of the geodesics of $\H^{n,1}$ or $\A^{n,1}$. Every lightlike or spacelike geodesic in $\H^{n,1}$ and $\A^{n,1}$ is topologically a line, therefore it has a countable number of lifts to $\AS^{n,1}$, On the other hand timelike geodesics $\H^{n,1}$ and $\A^{n,1}$ are topologically circles and are in bijection with timelike geodesics of $\AS^{n,1}$.\\

\section{Polarity in Anti-de Sitter space}
The quadratic form $q_{n,2}$ induces a polarity on the projective space $\mathbb{R}P^{n,1}$, namely the correspondance which associates to the projective subspace $P(W)$ the subspace $P(W^ \perp)$. In particular this induces a duality between spacelike totally geodesic subspaces of $\A^{n,1}$: the dual of a spacelike $k$-dimensional subspace is a $n-k+1$-dimensional subspace.
We indicate with $P_{[x]}=P(x^\perp)$ the hyperplane dual to a point $[x]\in \A^{n,1}$.
It can be checked that $P_{[x]}$ is the set of antipodal points to $[x]$ along timelike geodesics through $[x]$.
To some extent, this duality between points and planes lifts to the coverings of $\A^{n,1}$.\\

\noindent\textit{In the quadric model}. In $\H^{n,1}$ there are two dual planes associated to any point $x$: the sets 
\[
P_{x}^\pm=\{\exp_{x}(\pm(\pi/2)v)\,|\,q_{n,2}(v)=-1,\,v\text{ future-directed}\}~.
\]
Clearly $P_{x}^+$ and $P_{x}^{-} $ are antipodal and  $P_{-x}^\pm=P_{x}^\mp$. The planes $P_x^{\pm}$ disconnect $\H^{n,1}$ in two regions $U_x$ and
$U_{-x}$, where $U_{x}$ is the region containing $x$. They can be characterised by 
\[
U_{x}=\{y\in\H^{n,1}\,|\,\langle x, y\rangle_{n,1}<0\}.
\]

Spacelike and lightlike geodesics through $x$ do not exit $U_{x}$, 
while all the timelike geodesics through $x$ meet orthogonally $P^{\pm}_x$ and all pass through the point $-x$.
More precisely, a point $y\neq x$ is connected to $x$:
\begin{itemize}
\item by a spacelike geodesic if and only if $\langle x,y\rangle_{n,1}<-1$,
\item by a lightike geodesic if and only if $\langle x,y\rangle_{(n,1)}=-1$,
\item by a timelike geodesic if and only if 
$|\langle x,y\rangle_{(n,1)}|<1$.
\end{itemize}
An immediate consequence is that if $y$ is connected to $x$ by a spacelike geodesic, there is no geodesic joining $y$ to $-x$.
Hence the exponential map of $\H^{n,1}$ is not surjective. But as any point $y\in\H^{n,1}$ can be connected through a geodesic either to $x$ or to $-x$,
the exponential over $\A^{n,1}$ is surjective.\\

\noindent\textit{In the universal cover}.
Recall that the group of deck transformations for the covering $\AS^{n,1} \to \H^{n,1}$ is $\Z$, where a generator acts by translations of $2\pi$ in the $\R$ factor. Hence the  preimage of a spacelike plane $P\subset\A^{n,1}$ is the disjoint union of spacelike planes $(P^k)_{k\in\Z}$, enumerated so that the generator $\eta$ of $\Z$ acts by sending $P^k$ to $P^{k+1}$.
Moreover each connected component of $\A^{n,1}\setminus\bigcup_{k\in\Z}P^k$ is a fundamental  domain for the action of deck transformations of the covering $\AS^{n,1}\to\A^{n,1}$.

\chapter{Anti-de Sitter space in dimension (2+1)} \label{chapter:3}

\section{The $\PSL$-model}
In dimension three there is a special model which endows Anti-de Sitter space with a Lie group structure.
Consider the vector space $\mathcal{M}(2,\R)$ of $2 \times 2$ matrices with real entries and the quadratic form $q= -det$ that has signature $(2,2)$.
This gives an identification between $(\mathcal{M}(2,\R), -det)$ and $(\R^{2,2},q_{2,2})$, unique up to composition by elements in $O(2,2)$, and under this isomorphism $\H^{2,1}$ is identified with the Lie group $SL(2,\R)$.\\
$SL(2,\R)\times SL(2,\R)$ acts linearly on $\mathcal{M}(2,\R)$ by left and right multiplication:
\[
    (A,B) \cdot X := AXB^{-1}    
\]
preserving the quadratic form $q = -det$. Therefore it induces a representation
\[
    SL(2,\R)\times SL(2,\R) \to O(\mathcal{M}(2,\R),q).
\]
The kernel of this representation is given by $K=\{ (\1,\1),(-\1,-\1) \}$, and by dimensional argument it is the connected component of the identity:
\[
    \text{Isom}_0(\H^{2,1}) \cong SO_0(\mathcal{M}(2,\R),q) \cong SL(2,\R)\times SL(2,\R) / K.
\]
Using this model there is a natural identification of $\A^{2,1}$ with the Lie group $\PSL$ and
\[
    \text{Isom}_0(\A^{2,1}) \cong \PSL \times \PSL
\]
acting by left and right multiplication on $\PSL$.
The stabilizer of the identity in $\text{Isom}_0(\A^{2,1})$ is the diagonal subgroup $\Delta < \PSL \times \PSL$.\\

\section{The boundary of $\PSL$}
From the identification between $\A^{2,1}$ and $\PSL$ we obtain an identification of $\partial \A^{2,1}$ with the boundary of $\PSL$ into $P(\mathcal{M}(2,\R))$, which is the projectivization of rank 1 matrices:
\[
    \partial \A^{2,1} = \{ [X] \in P(\mathcal{M}(2,\R)) \ | \ rank(X)=1 \}.
\]
We have a homeomorphism
\begin{align}
    \partial \A^{2,1} \to \T \\
    [X] \to (\text{Im}X, \text{Ker}X)
\end{align}
equivariant under the action of $\PSL \times \PSL$, acting on $\partial\A^{2,1}$ as the natural extension of the action on $\A^{2,1}$, and by left multiplication on $\T$.\\
The equivariance is easily checked observing that $\text{Im}(AXB^{-1}) = A \cdot \text{Im}(X)$ and $\text{Ker}(AXB^{-1}) = B \cdot \text{Ker}(X)$.
\begin{lemma}
    The inversion map $\iota [X] = [X]^{-1}$ is a time-reversing isometry of $\A^{2,1}$ which induces the homeomorphism $(x,y) \to (y,x)$ on $\partial \A^{2,1} \cong \T$.
\end{lemma}

Using the half-plane model for the hyperbolic space one can identify $\partial \A^{2,1}$ with $\partial \H^2 \times \partial \H^2$ and we can interpret the convergence to $\partial \A^{2,1}$ in the following way:
\begin{lemma}
    A sequence $[X_n] \in \A^{2,1}$ converges to $(x,y) \in \partial \A^{2,1} \cong \T$ if and only if for every $p\in\H^2$, $X_n(p) \to x$ and $X_n^{-1}(p) \to y$.
\end{lemma}

\section{Geodesics in $\PSL$}
Let us start by understanding the geodesics through the identity. Using the Lie group structure of $\A^{2,1}$ it suffices to understand the one-parameter subgroup of $\A^{2,1}$
(the necessary tools in Lie groups theory are introduced in \cite{bonsanteseppi}). We get the following:
\begin{itemize}
    \item Timelike geodesics are, up to conjugacy, of the form
    \[ \begin{pmatrix}
        \text{cos}(t) & -\text{sin}(t) \\
        \text{sin}(t) & \text{cos}(t)
    \end{pmatrix} \]
    under the identification of $\PSL$ with $Isom(\H^2)$ they are elliptic one-parameter groups fixing a point in $\H^2$.
    \item Spacelike geodesics are, up to conjugacy
    \[ \begin{pmatrix}
        \text{cosh}(t) & \text{sinh}(t) \\
        \text{sinh}(t) & \text{cosh}(t)
    \end{pmatrix} \]
    these are hyperbolic one-parameter groups fixing two points in the boundary of $\H^2$
    \item Lightlike geodesics are the parabolic one-parameter groups conjugate to
    \[ \begin{pmatrix}
        1 & t \\
        0 & 1
    \end{pmatrix} \]
\end{itemize}
Using this description of timelike geodesics through $\1$, we can also interpret the duality in $\A^{2,1}$. Recalling that the dual plane of a point $A$ is the set of antipodal points along timelike geodesics through $A$, one sees that the dual plane of $\1$ consists of elliptic order-two isometries of $\H^2$. Equivalently, this is the set of (projective classes) of traceless matrices, that is (by the Cayley-Hamilton theorem)
\[
    P_\1 = \{ [J]\in\PSL \ | \ J^2 = -\1 \}.
\]
The boundary at infinity of $P_\1$ is made of traceless matrices of rank 1.\\
The stabilizer of $\1$ is the diagonal subgroup of $\PSL \times \PSL$, and it aso acts on the dual plane $P_\1$ by conjugation. Therefore the following statement is straightforward:
\begin{lemma}\label{lem:dual plane}
    The map from $\H^2$ to $P_\1$, sending $p\in\H^2$ to the elliptic order-two element in $\PSL$ fixing $p$, is a $\PSL$-equivariant isometry.
\end{lemma}

\noindent\textit{Timelike geodesics}.
To get a complete description of timelike geodesics it suffices to let the isometry group of $\A^{2,1}$ act on geodesics through the identity.
\begin{proposition}
    There is a homeomorphism between the space of timelike geodesics of $\A^{2,1}$ and $\H^2 \times \H^2$.
    The homeomorphism is equivariant for the action of $Isom_0(\A^{2,1}) \cong \PSL \times \PSL$.
\end{proposition}

\noindent\textit{Spacelike geodesics}.
Let $l$ be a geodesic of $\H^2$. The one-parameter group of hyperbolic transormation fixing $l$ as an oriented geodesic is a spacelike geodesic through the identity.
\begin{proposition}
    There is a homeomorphism between the space of oriented spacelike geodesics of $\A^{2,1}$ and the product of two copies of $\partial \H^2 \times \partial \H^2 \smallsetminus \Delta$. 
    The homeomorphism is equivariant for the action of $Isom_0(\A^{2,1}) \cong \PSL \times \PSL$.
\end{proposition}

\chapter{Globally hyperbolic spacetime}\label{chapter:4}

From now on we will only work with Lorentzian manifold of dimension $(2+1)$.
\section{Achronal and acausal set}

\begin{definition}
    A subset $X \subset \AS^{2,1} \cup \partial\AS^{2,1}$ is \textit{achronal} (resp. \textit{acausal}) is no pair of points in $X$ are connected by timelike (resp. causal) lines in $\AS^{2,1}$
\end{definition}
Consider the Poincaré model $\D\times\R$ of $\AS^{2,1}$ with the metric $g_{\mathbb{S}^2} - dt^2$. The following lemma gives a charecterization of achronal/acausal set:
\begin{lemma}
    A subset of $\AS^{2,1} \cup \partial\AS^{2,1}$ is achronal (resp. acausal) if and only if it is the graph of a function $\text{f} : D \to \R$ that is 1-Lipschitz (resp. strictly 1-Lipschitz) with respect to the distance induced by the hemispherical metric $g_{\mathbb{S}^2}$, where $D$ denotes the projection of $X$ to the $\D$ factor.
\end{lemma} 
\begin{definition}
    Given a surface $S$ and a Lorentzian manifold $(M,g)$, a $C^1$ immersion $\sigma : S \to M$ is \textit{spacelike} if the pull-back metric $\sigma^* g$ is a Riemannian metric. In this setting $\sigma(S)$ is said to be a \textit{spacelike surface}.
\end{definition}
A spacelike surface is locally acausal, if the immersion is proper we have the following global result:
\begin{lemma}
    Any properly embedded surface in $\AS^{2,1}$ is acausal.
\end{lemma}
\begin{definition}
    Let $X$ be an achronal set in $\AS^{2,1} \cup \partial\AS^{2,1}$, the \textit{invisible domain} of $X$ is the subset $\Omega(X) \subset \AS^{2,1}$ of points which are connected to $X$ by no causal path.
\end{definition}
\begin{definition}
    An \textit{achronal meridian} is a subset $\Lambda$ of $\partial\AS^{2,1}$ that os the graph of a 1-Lipschitz function $f: \partial\D\to\R$.
\end{definition}
The iportance of these definitions will be evident in the following sections.
The invisible domain of a achronal meridian will be fundamental tool in the study of hyperbolic spacetimes.

\section{Globally hyperbolic spacetime}
\begin{definition}
    Given an achronal subset $X$ in a Lorentzian manifold $(M,g)$, the \textit{domain of dependance} of $X$ is the set
    \[
        \mathcal{D}(X)= \{ p \in M \ | \ \text{every inextensible causal curve through p meet X} \}.
    \]
    If $\mathcal{D}(X)=M$ we say that M is  \textit{globally hyperbolic} whit \textit{Cauchy surface} $X$.
\end{definition}
Globally hyperbolic spacetime have strong geometric property we summarize in the following theorem:
\begin{theorem}
    Let $M$ be a globally hyperbolic spacetime, then
    \begin{itemize}
        \item Any two Cauchy surfaces are diffeomorphic.
        \item There  exists a submersion $\tau : M \to \R$ whose fibers are Cauchy surfaces.
        \item $M$ is diffeomorphic to $\Sigma \times \R$ where $\Sigma$ is a Cauchy surface.
    \end{itemize}
\end{theorem}
The aim of this section is to study maximal globally hyperbolic (MGH) Anti-de Sitter spacetimes containing a compact Cauchy surface of genus $n$ (in short we say that the globally hyperbolic spacetime has genus $n$). We will be interested mainly in the case $n\geq 2$ cause later on we will study MGH spacetime whose Cauchy surface is a closed hyperbolic surface.\\ 
\begin{observation}
    It can be shown that spacelike surfaces in $\A^{2,1}$ are properly embedded and diffeomorphic to $\R^2$.
    In particular there are no globally hyperbolic Anti-de Sitter spacetime of genus zero (\cite{bonsanteseppi}).
\end{observation}
\begin{proposition}\label{GH_geometry}
    Let $M$ be a globally hyperbolic Anti-de Sitter spacetime of genus $n\geq 1$. Then
    \begin{itemize}
        \item The developing map $dev: \widetilde{M} \to\A^{2,1}$ is injective.
        \item If $\Sigma$ is a Cauchy surface of $M$, then the image of dev is contained in $\Omega(\Lambda)$ where $\Lambda$ is the boundary of $dev(\widetilde{\Sigma})$.
        \item If $\rho : \pi_1(M) \to \text{Isom}(\A^{2,1})$ is the holonomy representetion, $\rho(\pi_1(M))$ acts freely and properly discontinuously on $\Omega(\Lambda)$ and $\Omega(\Lambda) / \rho(\pi_1(M))$ is a globally hyperbolic spacetime containig $M$.
    \end{itemize}
\end{proposition}
\begin{definition}
    A globally hyperbolic Anti-de Sitter spacetime $(M,g)$ is said to be \textit{maximal} if any isometric embedding of $(M,g)$ into a globally hyperbolic Anti-de Sitter spacetime $(M',g')$ which sends a Cauchy surface of $(M,g)$ to a Cauchy surface of $(M'.g')$ is surjective.
\end{definition}
Following from Proposition \ref{GH_geometry} we have:
\begin{corollary}
    A globally hyperbolic Anti-de Sitter spacetime $M$ is maximal if and only if $\widetilde{M}$ is isometric to the invisible domain of a proper achronal meridian in $\A^{2,1}$.
\end{corollary}

%sezione 5.4 e 5.5 su MGH di genere >1
\section{Genus $n\geq 2$}
In this section we eill classify MGH spacetime of genus $n\geq 2$.
Let $\Sigma_n$ be an oriented surface of genus $n\geq 2$.
\begin{definition}
    A representetion $\rho: \pi_1(\Sigma_n) \to \PSL$ is \textit{positive Fuchsian} if there is a $\rho$-equivariant orientation-preserving homeomorphism $\delta : \widetilde{\Sigma_n}\to\H^2$.
\end{definition}
The following classical result in Teich\"uller theory is essential for the construction of MGH spacetime of genus $n\geq 2$.
\begin{lemma}
    Given two positive Fuchsian representetion $\rho_l, \rho_r : \pi_1(\Sigma_n) \to \PSL$, any $(\rho_l, \rho_r)$-equivariant orientation-preserving homeomorphism of $\H^2$ extends continuously to an orientation-preserving homeomorphism of $\H^2\cup\S$. Moreover, its extension $\varphi : \S\to\S$ is the unique $(\rho_l, \rho_r)$-equivariant orientation-preserving homeomorphism of $\S$.
\end{lemma}
By $(\rho_l, \rho_r)$-equivariance of $\varphi$ we mean that for every $\gamma \in \pi_1(\Sigma_n)$:
\[
    \varphi \circ \rho_l(\gamma) = \rho_r(\gamma)\circ\varphi.
\]
Given two positive Fuchsian representetion $\rho_l, \rho_r : \pi_1(\Sigma_n) \to \PSL$ we will consider the associated representetion in Anti-de Sitter geometry given by
\[
    \rho = (\rho_l, \rho_r) : \pi_1(\Sigma_n) \to \text{Isom}_0(\A^{2,1}) \cong \PSL \times \PSL.
\]
In this setting we define $\Lambda(\rho)$ as the graph of $\varphi: \S\to\S$ defined by $\rho$, and $\Omega_\rho := \Omega(\Lambda(\rho))$ its invisible domain in $\A^{2,1}$.
\begin{proposition}
    The domain $\Omega_\rho$ is invariant under the isometric action of $\pi(\Sigma_n)$ on $\A^{2,1}$ induced by $\rho$. Moreover $\pi_1(\Sigma_n)$ acts freely and properly discontinuously on $\Omega_\rho$ and the quotient is a MGH spacetime of genus $n$ and holonomy $\rho$.
\end{proposition}
$\Lambda(\rho)$ is also the unique achronal meridian in $\partial\A^{2,1}$ invariant under the action of $\pi_1(\Sigma_n)$ induced by $\rho$ making $M_\rho = \Omega_\rho / \rho(\pi_1(\Sigma_n))$ the unique spacetime with holonomy $\rho$.\\
One can also prove that if $M$ is a time-oriented MGH spacetime it is of the form $M_\rho$. More precisely:
\begin{proposition}
    Let $M$ be an oriented, time-oriented, globally hyperbolic spacetime of genus $n\geq 2$ and let us endow a Cauchy $\Sigma$ with the orientation induced by the future normal vector. Then the left and right components of the holonomy $\rho=(\rho_l,\rho_r): \pi_1(\Sigma)\to\PSL\times\PSL$ are positive Fuchsian representetion.
\end{proposition}


We conclude by stating the classification result. Let the \emph{deformation space} of MGH spacetimes of genus $n$ be:
$$\mathcal{MGH}(\Sigma_n)=\{g\text{ MGH AdS metric on }\Sigma_r\times\R\}/\mathrm{Diff}_0(\Sigma_n\times\R)~,$$
where the group of diffeomorphisms isotopic to the identity acts by pull-back. The holonomy map takes value in the space of representations of $\pi_1(\Sigma_r)$ into $\PSL\times\PSL$ and is well-defined on the quotient $\mathcal{MGH}(\Sigma_r)$.
We proved that the left and right components of the holonomy of elements of $\mathcal{MGH}(\Sigma_n)$ are positive Fuchsian representations, and the space of these representations up to conjugacy is identified with the Teichm\"uller space of $\Sigma_n$
\[
    \mathcal{T}(\Sigma_n)\cong\{\rho:\pi_1(\Sigma_n)\to\PSL\text{ positive Fuchsian representations}\}/\PSL~.
\]
Therefore the holonomy map can be considered as a map 
from $\mathcal{MGH}(\Sigma_n)$ with values in $\mathcal{T}(\Sigma_n)\times\mathcal{T}(\Sigma_n)$.
Hence, we can summarize this section with the following theorem of Mess.

\begin{theorem} \label{thm:classification rgeq2}
The holonomy map $$\rho:\mathcal{MGH}(\Sigma_n)\to\mathcal{T}(\Sigma_n)\times\mathcal{T}(\Sigma_n)$$ is a homeomorphism.
\end{theorem}

\chapter{Gauss map and spacelike immersion}

\section{Spacelike surface in $\A^{2,1}$}
In this section we will briefly talk about geometric property of immersed spacelike surfaces, that is the analogous of the theory for the Euclidean space.\\
Let us denote with $\nabla$ the Levi-Civita connection of the Lorentzian metric of $\A^{2,1}$. Given a spacelike immersion $\sigma: S\to\A^{2,1}$ the pull-back metric $\I = \sigma^*(g_{\A^{2,1}})$ is called \textit{first fundamental form} of $\sigma$.
The tangent bundle $TS$ is naturally identified to a subbundle of the pull-back $\sigma^*(TM)$, therefore we can define its normal bundle $N_\sigma$. Using the decomposition
\[
    \sigma^* TM = TS \oplus N_\sigma,
\]
the pull-back of the Levi-Civita connection $\nabla$, restricted to sections tangent to $S$ splits as the sum of the Levi-Civita connection of the first fundamental form $\I$ and a symmetric $2-form$ with value in $N_\sigma$. Given the time-orientability, we take the future-directed unit normal vector field $\nu$ of $S$. The Levi-Civita connection $\nabla^\I$ of the first fundamental form $\I$ of S is defined by the relation:
\[
    \nabla_V W= \nabla^\I_V W + \II(V,W)\nu,
\] 
where the $(2,0)$-tensor $\II$ is called \textit{second fundamental form} of $\sigma$. The \textit{shape  operator} of $\sigma$ is a $(1,1)$-tensor $B$ defined as
\[
    B(V) = - \nabla_V N.
\]
The shape operator is related to the second fundamental form by
\[
    \II(V,W) = \I(B(V),W).
\]
In particular, $B$ is diagonalisable and its eigenvalues are colled principal curvatures.\\
The first and second fundamental form of an immersion $\sigma$ satisfy constraint equations known as the \textit{Gauss-Codazzi equations}. More precisely the Gauss equation consists of the identity
\[
    K_\I = -1 - \text{det}B.
\]
The Codazzi equation tells us that
\[
    d^{\nabla^\I}B=0,
\]
where $d^{\nabla^\I}$ is the operator defined by
\[
    (d^{\nabla^\I}B)(V,W) = \nabla^\I_V(B(W)) - \nabla^\I_W(B(V)) - B([V,W]).
\]
Sometimes is also written in the equivalent form 
\[
    (\nabla^\I_V \II)(W,U) = (\nabla^\I_W \II)(V,U).
\]
As in the Euclidean space, the embedding data $I$ and $\II$ of a simply connected surface determines the immersion uniquely up to isometries of $\A^{2,1}$
\begin{theorem}\label{thm:immersion of simply connected surface}
    Let $S$ be a simply connected surface, let $\I$ be a Riemannian metric on $S$ and $\II$ be a symmetric $(2,0)$-tensor on $S$. If $\I$ and $\II$ satisfy the Gauss-Codazzi equations, then there exists a spacelike immersion $\sigma:S \to\A^{2,1}$ having $\I$ and $\II$ as first and second fundamental form. Moreover if $\sigma$ and $\sigma'$ are two such immersions, then there exists a time-preserving isometry $\varphi$ of $\A^{2,1}$ such that $\sigma' = \varphi \circ \sigma$.
\end{theorem}

\section{Germs of spacelike immersions}
Let us now consider che case in which $S$ is an oriented surface, not necessarily simply connected. Let $\sigma: S \to(M,g)$ a spacelike immersion where $(M,g)$ is an oriented AdS spacetime.
As in the previous section, we can associate to $\sigma$ the pair $(\I,\II)$ of first and second  fundamental form, where $\II$ is  computed with respect to the unit normal vector $\nu$ of $\sigma$.
We will always assume that the orientation of $S$ and $\nu$ are compatible with the one on $M$.\\
We want to prove that the pair $(\I,\II)$ depends only on the \textit{germ} of $\sigma$, which is defined as follow:
\begin{definition}
    A \textit{germ} of a spacelike immersion of $S$ into an AdS spacetime is an equivalent class of spacelike immersions $\sigma:S \to(M,g)$ by the following relation: $\sigma:S \to(M,g)$ and $\sigma':S \to(M',g')$ are equivalent if there exist open subsets $U \supset \sigma(S)$ in $M$ and $U' \supset \sigma'(S)$ in $M'$ and a time-preserving isometry $f:(U,g)\to(U',g')$ such that $\sigma' = f \circ \sigma$.
\end{definition}
Given a pair $(\I,\II)$ on $S$ and $\pi: \widetilde{S}\to S$ a universal cover, one can take the pair $(\pi^*\I, \pi^*\II)$ on $\widetilde{S}$ that clearly satisfy the Gauss-Codazzi equations. By the existence part of Theorem \ref{thm:immersion of simply connected surface} there exists a spacelike immersion $\widetilde{\sigma}: \widetilde{S} \to\A^{2,1}$ having immersion data $(\pi^*\I, \pi^*\II)$. As a concequence of the uniqueness part of Theorem \ref{thm:immersion of simply connected surface} we have that associeted to $\widetilde{\sigma}$ there is a map $\rho:\pi_1(S) \to \text{Isom}_0(\A^{2,1})$ such that for every $\gamma \in \pi_1(S)$, $\widetilde{\sigma} \circ \gamma = \rho(\gamma) \circ \widetilde{\sigma}$. Moreover changing $\widetilde{\sigma}$ by post-composition with an isometry $f$ of $\A^{2,1}$ has the effect of conjugating $\rho$ by $f$.\\
Given $\sigma:S \to \A^{2,1}$ one can extend the immersion to an immersion of an open neighbourhood of $S \times\{0\}$ in $S\times\R$ into $\A^{2,1}$, by mapping $(x,t)$ to the point $\gamma(t)$ on the timelike geodesic $\gamma$ such that $\gamma(0)=\sigma(p)$ and $\gamma'(0)$ is the future normal vector of $\sigma$ at $x$.
\begin{lemma}\label{lem:tub metric}
    Given a spacelike immersion $\sigma:S\to \A^{2,1}$, the pull-back metric by means of the map $(p,t) \to exp_{\sigma(x)}(t\nu(x))$ has the expression:
    \begin{equation}\label{eq:tub metric}
        -dt^2 + \text{cos}^2(t)\I + 2\text{cos}(t)\text{sin}(t)\II + \text{sin}^2(t)\III,
    \end{equation}
    where $\I$, $\II$, $\III$ are the first, second and third fundamental form of $\sigma$ and third fundamental form is defined as $\III(\cdot,\cdot)=\I(B(\cdot),B(\cdot))$.
\end{lemma}
Therefore, given a pair $(\I,\II)$, the expression \ref{eq:tub metric} provides a Lorentzian metric of constant curvature $-1$ on an open neighbourhood of $S \times\{0\}$ in $S\times\R$ into $\A^{2,1}$, and thus a germ of immersion of $S$ into and AdS spacetime with immersion data $(\I,\II)$. We conclude the above discussion with the following:
\begin{proposition}\label{thm:immersion data classification}
    Given a surface $S$, there are natural identidications between the following spaces:
    \begin{itemize}
        \item The space of pairs $(\I,\II)$ on $S$ which are solutions of the Gauss-Codazzi equations,
        \item The space of germs of spacelike immersions of $S$ into Anti-de Sitter manifolds,
        \item The space of spacelike immersions of $\widetilde{S}$ into $\A^{2,1}$, equivariant with respect to a representation $\rho:\pi_1(S) \to \text{Isom}_0(\A^{2,1})$, up to the action of $\text{Isom}_0\A^{2,1}$ by post-composition.
    \end{itemize}
\end{proposition}
When $S$ is a closed surface the equivariant immersion $\widetilde{\sigma}$ is necessarily an embedding, which can be extended to an embedding of $\widetilde{S}\times\R$ into a domain of dependence in $\A^{2,1}$. The representation $\rho:\pi_1(S) \to \PSL \times \PSL$ coincides with the holonomy of the MGH spacetime $(M,g) \cong \widetilde{S}\times\R$, after identifying $\pi_1(S)$ with $\pi_1(M)$, and therefore $\rho$ consists of a pair $(\rho_l,\rho_r)$ of positive Fuchsian representation.\\
Therefore, the embedding data $(\I,\II)$ permit to recover the pair of elements in $\mathcal{T}(S)\times\mathcal{T}(S)$ which parametrizes MGH Anti-de Sitter manifolds with compact Cauchy surfaces.

\section{Gauss map}
We can now define the Gauss map for spacelike surfaces in $\A^{2,1}$. Recall that the space of timelike geodesics in $\A^{2,1}$ is identified with $\H^2\times\H^2$, where the identification maps a geodesic of the form
\[
    L_{(p,q)}=\{ X \in\PSL \ | \ X \cdot q = p \}
\]
to the pair $(p,q) \in \H^2\times\H^2$. 
\begin{definition}
    Let $\sigma :S \to\A^{2,1}$ a spacelike immersion. The \textit{Gauss map} $G_\sigma : S \to \H^2 \times \H^2$ is defined as $G_\sigma(x) = (p,q)$ such that $L_{(p,q)}$ is the timelike orthogonal to $\text{Im}(d_x\sigma)$ at $\sigma(x)$.\\
    The components of the Gauss map, denoted by $\Pi_l, \Pi_r : S \to \A^{2,1}$, are called \textit{left} and \textit{right projections}. 
\end{definition}
The Gauss map $G_\sigma$ is natural with respect to the action of the isometry group, meaning that $G_{f\circ\sigma} = f \cdot G_\sigma$ for every $f \in\text{Isom}_0(\A^{2,1}) = \PSL\times\PSL$. The Gauss map is also invariant by reparametrization, in the sense that $G_{\sigma \circ \phi} = G_\sigma \circ \phi$ for a diffeomorphism $\phi:S'\to S$. Hence it makes sense to talk about the Gauss map of a spacelike surface in $\A^{2,1}$.
\begin{proposition}\label{prop:left right pull-back metric}
    Let $\sigma:S\to\A^{2,1}$ be a spacelike immersion, let $\Pi_l,\Pi_r:S\to\H^2$ be the left and right projections and let $g_{\H^2}$ be the hyperbolic metric. Then 
    \begin{equation}
        \Pi_l^*g_{\H^2}=I((\text{id}-\mathcal{J}B)\cdot,(\text{id}-\mathcal{J}B)\cdot)\;\text{and}\;\Pi_r^*(g_{\H^2})=I((\text{id}+\mathcal{J}B)\cdot,(\text{id}+\mathcal{J}B)\cdot),
    \end{equation}
    where $I$ is the first fundamental form of $\sigma$, $\mathcal{J}$ its associated almost-complex structure, and $B$ the shape operator.
\end{proposition}
We collect here some consequences and remarks around Proposition \ref{prop:left right pull-back metric}.
\begin{itemize}[leftmargin=0.5cm]
\item Proposition \ref{prop:left right pull-back metric} shows that the differential of the left and right projections essentially has the expression 
\[
    d_x\sigma\circ(B\pm\JJ)~,
\]
up to post-composing with an isometry sending the image of $d_x\sigma$ to a fixed copy of $\H^2$. Since $B$ is $\I$-symmetric, $\JJ\circ B$ is traceless, and therefore 
\begin{equation}\label{eq:deteminant differential projections}
\det(B\pm\JJ)=1+\det B=-K_\I~.
\end{equation}
This shows that $\Pi_l$ is a local diffeomorphism at a point $x$ if and only if $\Pi_r$ is, which is the case if and only if the intrinsic curvature of $\I$ at $x$ is different from $0$.
\item Since the trace of $B\pm\JJ$ equals $2$, the differentials of $\Pi_l$ and $\Pi_r$ have either rank 2 or rank 1. (In fact, by \eqref{eq:deteminant differential projections}, when the differential of $\Pi_l$ has rank 1, the same holds for the differential of $\Pi_r$.) Hence the differential of the Gauss map $G:S\to\H^2\times\H^2$ is always non-singular.
\item If an immersed surface has the property that the curvature of the first fundamental form never vanishes, and if moreover $\Pi_l$ and $\Pi_r$ are globally injective, then the image of $G$ is the graph of a  diffeomorphism $F_\sigma$ between two subsets of $\H^2$,  called the \emph{associated map}. From Equation \eqref{eq:deteminant differential projections}, the Jacobians of $\Pi_l$ and $\Pi_r$ are equal, hence the associated map is area-preserving.
When $\Pi_l$ and $\Pi_r$ are only locally injective, but not globally, we still obtain an area-preserving local diffeomorphism $F_\sigma$ which is now defined between two hyperbolic surfaces, not globally isometric to subsets of $\H^2$.
\item More generally, as a consequence of the previous points, the image of $G$ is always a \emph{Lagrangian submanifold} in $\H^2\times\H^2$ with respect to the symplectic form 
\begin{equation} \label{eq symplectic form}
\Omega=\pi_l^*\omega_{\H^2}-\pi_r^*\omega_{\H^2}~,
\end{equation}
where $\omega_{\H^2}$ is the hyperbolic area form. 
This result has been proved in several works with different methods: see \cite{bonsante2017equivariant}, \cite{Seppi_2017}. Moreover the Lagrangian condition is \emph{locally} the only obstruction to inverting this construction, that is, to realizing an immersed surfaces in $\H^2\times\H^2$ locally as the image of the Gauss map of a spacelike immersion in $\A^{2,1}$.
\item Given a spacelike immersion $\sigma$,  the normal evolution of $\sigma$ is defined as 
\[
    \sigma_t(x)=\exp_{\sigma(x)}(t\nu(x))~,
\]
where $\nu$ is the future unit normal vector field. In general $\sigma_t$ may fail to be an immersion for $|t|$ large. When it is an immersion, the computation of the metric in Lemma \ref{lem:tub metric} shows that the image of $\sigma_t$ at $x$ is orthogonal to the geodesic $\gamma(t)=\exp_{\sigma(x)}(t\nu(x))$. In other words, the Gauss map of $\sigma_t$ is equal to the Gauss map of $\sigma$. Hence with respect to the previous point, given a spacelike immersion which have the same  Lagrangian submanifold of $\H^2\times\H^2$ as Gauss map image.
\end{itemize}

\chapter{Minimal Lagrangian diffeomorphism}
In this last chapter we want to recover the existence and uniqueness of minimal Lagrangian using Anti-de Sitter geometry.
\section{Foliations}
A smoothly embedded spacelike surface $S$ has \textit{constant mean curvature} if the trace of the shape operator $B$ is constant. A \textit{maximal surface} is a surface $S$ whose mean curvature is constantly equal to zero.\\
\begin{theorem}\label{thm:CMC_foliation}
    Every MGH Anti-de Sitter manifold with compact Cauchy surface is uniquely foliated by closed CMC surfaces, where the mean curvature $H$ varies in $(-\infty,\infty)$.
\end{theorem}
A smoothly embedded spacelike surface $S$ has \textit{constant Gaussian curvature} if $\text{det}B$ is constant. We consider the case in which the Gaussian curvature $K$ is positive. A CGC surface results to be either locally convex or locally concave, where this distinction depends on the time orientation. 
\begin{theorem}\label{thm:CGC_foliation}
    Let $M$ be a MGH Anti-de Sitter manifold with compact Cauchy surface. Then each connected component of $M \setminus C(M)$ is uniquely foliated bt closed CGC surfaces, where the Gaussian curvature $K$ varies in $(0,\infty)$.
\end{theorem}
The proofs of theorems \ref{thm:CMC_foliation} and \ref{thm:CGC_foliation} can be found in \cite{barbot2004constant} and \cite{barbot2008prescribing} respectively. Moreover in \cite{barbot2008prescribing} one can find what we mean by $C(M)$, namely the convex core of $M$.
\begin{observation}
    For every $H$ the CMC is surface is unique, as an application of the maximum principle. Moreover the CMC foliation is determined by a time function $\tau:M\to\R$, which is a function stricly increasing along future-directed causal curve whose fibers are the CMC surfaces. Analogously, for every $K$ the CGC surface is unique in its connected component, and the foliation is determined by a time function whose fibers are CGC surfaces.
\end{observation}
Using the normal evolution
\[
    \sigma_t(x) = \text{exp}_{\sigma(x)} (t\nu(x))
\]
one can also find a relation between CMC and CGC surfaces:
\begin{proposition}
    Let $\sigma: S \to\A^{2,1}$ be an immersion of constant Gaussian curvature $K>0$. Then the normal evolution $\sigma_{t_K}$ on the convex side of $\sigma$, for time $t_K=\text{arctan}(K^{1/2})$, is an immersion of constant mean curvature $H=K^{-1/2}(K-1)$.
\end{proposition}
Using this construction and the uniqueness of CMC surfaces, one find that each surface $\Sigma_H$ of constant mean curvature $H$ has two equidistant surfaces of constant Gaussian curvature $K_+$ and $K_-$, one convex in the past of $\Sigma_H$, the othe concave in its future.

\section{Minimal Lagrangian diffeomorphism}

\begin{definition}
    Let $h$ and $h'$ be two hyperbolic metrics on the closed surface $S$. A diffeomorphism $f: (S,h) \to (S,h')$ is \textit{minimal Lagrangian} if its graph is a minimal Lagrangian submanifold of $S\times S$ with respect to the product metric $h \oplus h'$ and the symplectic form $\pi_l^*\omega_h - \pi_r^* \omega_{h'}$. 
\end{definition}
Given a maximal surface $\Sigma_0$ in a maximal globally hyperbolic spacetime $(M,g)$, with compact Cauchy surface $\Sigma$, we claim that the associated map $f_0$ is a minimal Lagrangian map from $(\Sigma,h)$ to $(\Sigma,h')$, where $h$ and $h'$ are the quotient metrics induced in $\H^2/\rho_l(\pi_1\Sigma)$ and $\H^2/\rho_l(\pi_1\Sigma)$.\\
We have already discussed the Lagrangian condition, which amounts to $f_0$ being area-preserving and is always verified by the Gauss map image.
We shall show that the Gauss map is conformal and harmonic, which implies that its image is a minimal surface. 
By Proposition \ref{prop:left right pull-back metric} the pull-back of the product Riemannian metric has the expression $2(\I+\III)$.
When the trace of $B$ vanishes identically $\III=-(\det B)\I$ showing conformality. Also, the projections are local diffeomorphisms since $\Sigma_0$ is obtained as the equidistant surface from a convex surface (of intrinsic curvature $-2$), the projections are always local diffeomorphisms on convex surfaces, and its Gauss map coincides with that of $\Sigma_0$. Therefore, by a topological argument, the projections are then global diffeomorphisms from $\Sigma_0$ to $(\Sigma,h)$ and $(\Sigma,h')$.
The harmonicity of the Gauss map is equivalent to the harmonicity of each projection. Since the notion of harmonic map between Riemannian surfaces only depends on the conformal structure on the source, it suffices to show that 
$$\Pi_l:(\Sigma_0,\I)\to (\Sigma_0,\I((id-\JJ\circ B)\cdot,(id-\JJ\circ B)\cdot))$$ is harmonic, and the same for $\Pi_r$. To see this,  we can rewrite the target metric as $(\I+\III)-2\I((\JJ\circ B)\cdot,\cdot)$. We have used that $B$ is traceless and thus $\JJ\circ B$ is $\I$-symmetric. Together with the Codazzi property of $\JJ\circ B$, this also implies that  $2\I((\JJ\circ B)\cdot,\cdot)$ is the real part of a holomorphic quadratic differential, in light of the following well-known fact, see
 
\begin{lemma}\label{lemma hopf}
Given a Riemannian metric $g$ on a surface and a $(1,1)$-tensor $A$, $A$ is traceless if and only if $g(A\cdot,\cdot)$ is the real part of a quadratic differential for the conformal structure of $g$. Moreover the quadratic differential is holomorphic if and only if $A$ is $g$-Codazzi.
\end{lemma}
 
Therefore $\Pi_l$ is harmonic. The same proof clearly holds for $\Pi_r$.\\ This construction can actually be reversed, in the sense that every minimal Lagrangian map can be realized as the  map associated with a maximal surface. This permits to reprove the following theorem of existence and uniqueness of minimal Lagrangian diffeomorphisms in a given isotopy class:

\begin{theorem}
Given a closed surface $\Sigma$ and two hyperbolic metrics $h$ and $h'$ on $\Sigma$, there exists a unique minimal Lagrangian diffeomorphism $f_0:(\Sigma,h)\to (\Sigma,h')$ isotopic to the identity.
\end{theorem}

\printbibliography
\end{document}