\chapter{Gauss map and spacelike immersion}

section{Spacelike surface}
In this section we will briefly talk about geometric property of immersed spacelike surfaces, that is the analogous of the theory for the Euclidean space.\\
Let us denote with $\nabla$ the Levi-Civita connection of the Lorentzian metric of $\A^{2,1}$. Given a spacelike immersion $\sigma: S\to\A^{2,1}$ the pull-back metric $\I = \sigma^*(g_{\A^{2,1}})$ is called \textit{first fundamental form} of $\sigma$.
The tangent bundle $TS$ is naturally identified to a subbundle of the pull-back $\sigma^*(TM)$, therefore we can define its normal bundle $N_\sigma$. Using the decomposition
\[
    \sigma^* TM = TS \oplus N_\sigma,
\]
the pull-back of the Levi-Civita connection $\nabla$, restricted to sections tangent to $S$ splits as the sum of the Levi-Civita connection of the first fundamental form $\I$ and a symmetric $2-form$ with value in $N_\sigma$. Given the time-orientability, we take the future-directed unit normal vector field $\nu$ of $S$. The Levi-Civita connection $\nabla^\I$ of the first fundamental form $\I$ of S is defined by the relation:
\[
    \nabla_V W= \nabla^\I_V W + \II(V,W)\nu,
\] 
where the $(2,0)$-tensor $\II$ is called \textit{second fundamental form} of $\sigma$. The \textit{shape  operator} of $\sigma$ is a $(1,1)$-tensor $B$ defined as
\[
    B(V) = - \nabla_V N.
\]
The shape operator is related to the second fundamental form by
\[
    \II(V,W) = \I(B(V),W).
\]
In particular, $B$ is diagonalisable and its eigenvalues are colled principal curvatures.\\
The first and second fundamental form of an immersion $\sigma$ satisfy constraint equations known as the \textit{Gauss-Codazzi equations}. More precisely the Gauss equation consists of the identity
\[
    K_\I = -1 - \text{det}B.
\]
The Codazzi equation tells us that
\[
    d^{nabla^\I}B=0,
\]
where $d^{nabla^\I}$ is the operator defined by
\[
    (d^{nabla^\I}B)(V,W) = \nabla^\I_V(B(W)) - \nabla^\I_W(B(V)) - B([V,W]).
\]
Sometimes is also written in the equivalent form 
\[
    (\nabla^\I_V \II)(W,U) = (\nabla^\I_W \II)(V,U).
\]
As in the Euclidean space, the embedding data $I$ and $\II$ of a simply connected surface determines the immersion uniquely up to isometries of $\A^{2,1}$
\begin{theorem}\label{thm:immersion of simply connected surface}
    Let $S$ be a simply connected surface, let $\I$ be a Riemannian metric on $S$ and $\II$ be a symmetric $(2,0)$-tensor on $S$. If $\I$ and $\II$ satisfy the Gauss-Codazzi equations, then there exists a spacelike immersion $\sigma:S \to\A^{2,1}$ having $\I$ and $\II$ as first and second fundamental form. Moreover if $\sigma$ and $\sigma'$ are two such immersions, then there exists a time-preserving isometry $\varphi$ of $\A^{2,1}$ such that $\sigma' = \varphi \circ \sigma$.
\end{theorem}

\section{Germs of spacelike immersions}
Let us now consider che case in which $S$ is an oriented surface, not necessarily simply connected. Let $\sigma: S \to(M,g)$ a spacelike immersion where $(M,g)$ is an oriented AdS spacetime.
As in the previous section, we can associate to $\sigma$ the pair $(\I,\II)$ of first and second  fundamental form, where $\II$ is  computed with respect to the unit normal vector $\nu$ of $\sigma$.
We will always assume that the orientation of $S$ and $\nu$ are compatible with the one on $M$.\\
We want to prove that the pair $(\I,\II)$ depends only on the \textit{germ} of $\sigma$, which is defined as follow:
\begin{definition}
    A \textit{germ} of a spacelike immersion of $S$ into an AdS spacetime is an equivalent class of spacelike immersions $\sigma:S \to(M,g)$ by the following relation: $\sigma:S \to(M,g)$ and $\sigma':S \to(M',g')$ are equivalent if there exist open subsets $U$ in $M$ and $U'$ in $M'$ and a time-preserving isometry $f:(U,g)\to(U',g')$ such that $\sigma' = f \circ \sigma$.
\end{definition}
\red{forse non è ovvio che è una rel. di equivalenza}\\
Given a pair $(\I,\II)$ on $S$ and $\pi: \widetilde{S}\to S$ a universal cover, one can take the pair $(\pi^*\I, \pi^*\II)$ on $\widetilde{S}$ that clearly satisfy the Gauss-Codazzi equations. By the existence part of Theorem \ref{thm:immersion of simply connected surface} there exists a spacelike immersion $\widetilde{\sigma}: \widetilde{S} \to\A^{2,1}$ having immersion data $(\pi^*\I, \pi^*\II)$. As a concequence of the uniqueness part of Theorem \ref{thm:immersion of simply connected surface} we have that associeted to $\widetilde{\sigma}$ there is a map $\rho:\pi_1(S) \to \text{Isom}_0(\A^{2,1})$ such that for every $\gamma \in \pi_1(S)$, $\widetilde{\sigma} \circ \gamma = \rho(\gamma) \circ \widetilde{\sigma}$. Moreover changing $\widetilde{\sigma}$ by post-composition with an isometry $f$ of $\A^{2,1}$ has the effect of conjugating $\rho$ by $f$.\\
The immersion $\sigma$ can be extended to an immersion of an open neighbourhood of $S \times\{0\}$ in $S\times\R$ into $\A^{2,1}$, by mapping $(x,t)$ to the point $\gamma(t)$ on the timelike geodesic $\gamma$ such that $\gamma(0)=\sigma(p)$ and $\gamma'(0)$ is the future normal vector of $\sigma$ at $x$.
\begin{lemma}\label{lem:tub metric}
    Given a spacelike immersion $\sigma:S\to \A^{2,1}$, the pull-back metric by means of the map $(p,t) \to exp_{\sigma(x)}(t\nu(x))$ has the expression:
    \begin{equation}\label{eq:tub metric}
        -dt^2 + \text{cos}^2(t)\I + 2\text{cos}(t)\text{sin}(t)\II + \text{sin}^2(t)\III,
    \end{equation}
    where $\I$, $\II$, $\III$ are the first, second and third fundamental form of $\sigma$ and third fundamental form is defined as $\III(\cdot,\cdot)=\I(B(\cdot),B(\cdot))$.
\end{lemma}
\begin{proof}
    \red{mettere proof}
\end{proof}
Therefore, given a pair $(\I,\II)$, the expression \ref{eq:tub metric} provides a Lorentzian metric of constant curvature $-1$ on an open neighbourhood of $S \times\{0\}$ in $S\times\R$ into $\A^{2,1}$, and thus a germ of immersion of $S$ into and AdS spacetime with immersion data $(\I,\II)$. We conclude the above discussion with the following:
\begin{proposition}\label{thm:immersion data classification}
    Given a surface $S$, there are natural identidications between the following spaces:
    \begin{itemize}
        \item The space of pairs $(\I,\II)$ on $S$ which are solutions of the Gauss-Codazzi equations,
        \item The space of germs of spacelike immersions of $S$ into Anti-de Sitter manifolds,
        \item The space of spacelike immersions of $\widetilde{S}$ into $\A^{2,1}$, equivariant with respect to a representation $\rho:\pi_1(S) \to \text{Isom}_0(\A^{2,1})$, up to the action of $\text{Isom}_0\A^{2,1}$ by post-composition.
    \end{itemize}
\end{proposition}
When $S$ is a closed surface the equivariant immersion $\widetilde{sigma}$ is necessarily an embedding, which can be extended to an embedding of $\widetilde{S}\times\R$ into a domain of dependence in $\A^{2,1}$. The representation $\rho:\pi_1(S) \to \PSL \times \PSL$ coincides with the holonomy of the MGH spacetime $(M,g) \cong \widetilde{S}\times\R$, after identifying $\pi_1(S)$ with $\pi_1(M)$, and therefore $\rho$ consists of a pair $(\rho_l,\rho_r)$ of positive Fuchsian representation.

\section{Gauss map}
We can now define the Gauss map for spacelike surfaces in $\A^{2,1}$. Recall that the space of timelike geodesics in $\A^{2,1}$ is identified with $\H^2\times\H^2$, where the identification maps a geodesic of the form
\[
    L_{(p,q)}=\{ X \in\PSL \ | \ X \cdot q = p \}
\]
to the pair $(p,q) \in \H^2\times\H^2$. 
\begin{definition}
    Let $\sigma :S \to\A^{2,1}$ a spacelike immersion. The \textit{Gauss map} $G_\sigma : S \to \H^2 \times \H^2$ is defined as $G_\sigma(x) = (p,q)$ such that $L_{(p,q)}$ is the timelike orthogonal to $\text{Im}(d_x\sigma)$ at $\sigma(x)$.
\end{definition}
The Gauss map $G_\sigma$ is natura with respect to the action of the isometry group, meaning that $G_{f\circ\sigma} = f \cdot G_\sigma$ for every $f \in\text{Isom}_0(\A^{2,1}) = \PSL\times\PSL$. The Gauss map is also invariant by reparametrization, in the sense that $G_{\sigma \circ \phi} = G_\sigma \circ \phi$ for a diffeomorphism $\phi:S'\to S$. Hence it makes sense to talk about the Gauss map of a spacelike surface in $\A^{2,1}$.