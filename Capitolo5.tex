\chapter{Gauss map and spacelike immersion}

\red{Numerare equazioni, aggiungiere riferimenti}
\section{Spacelike surface in $\A^{2,1}$}
In this section we will briefly talk about geometric property of immersed spacelike surfaces, that is the analogous of the theory for the Euclidean space.\\
Let us denote with $\nabla$ the Levi-Civita connection of the Lorentzian metric of $\A^{2,1}$. Given a spacelike immersion $\sigma: S\to\A^{2,1}$ the pull-back metric $\I = \sigma^*(g_{\A^{2,1}})$ is called \textit{first fundamental form} of $\sigma$.
The tangent bundle $TS$ is naturally identified to a subbundle of the pull-back $\sigma^*(TM)$, therefore we can define its normal bundle $N_\sigma$. Using the decomposition
\[
    \sigma^* TM = TS \oplus N_\sigma,
\]
the pull-back of the Levi-Civita connection $\nabla$, restricted to sections tangent to $S$ splits as the sum of the Levi-Civita connection of the first fundamental form $\I$ and a symmetric $2-form$ with value in $N_\sigma$. Given the time-orientability, we take the future-directed unit normal vector field $\nu$ of $S$. The Levi-Civita connection $\nabla^\I$ of the first fundamental form $\I$ of S is defined by the relation:
\[
    \nabla_V W= \nabla^\I_V W + \II(V,W)\nu,
\] 
where the $(2,0)$-tensor $\II$ is called \textit{second fundamental form} of $\sigma$. The \textit{shape  operator} of $\sigma$ is a $(1,1)$-tensor $B$ defined as
\[
    B(V) = - \nabla_V N.
\]
The shape operator is related to the second fundamental form by
\[
    \II(V,W) = \I(B(V),W).
\]
In particular, $B$ is diagonalisable and its eigenvalues are colled principal curvatures.\\
The first and second fundamental form of an immersion $\sigma$ satisfy constraint equations known as the \textit{Gauss-Codazzi equations}. More precisely the Gauss equation consists of the identity
\[
    K_\I = -1 - \text{det}B.
\]
The Codazzi equation tells us that
\[
    d^{\nabla^\I}B=0,
\]
where $d^{\nabla^\I}$ is the operator defined by
\[
    (d^{\nabla^\I}B)(V,W) = \nabla^\I_V(B(W)) - \nabla^\I_W(B(V)) - B([V,W]).
\]
Sometimes is also written in the equivalent form 
\[
    (\nabla^\I_V \II)(W,U) = (\nabla^\I_W \II)(V,U).
\]
As in the Euclidean space, the embedding data $I$ and $\II$ of a simply connected surface determines the immersion uniquely up to isometries of $\A^{2,1}$
\begin{theorem}\label{thm:immersion of simply connected surface}
    Let $S$ be a simply connected surface, let $\I$ be a Riemannian metric on $S$ and $\II$ be a symmetric $(2,0)$-tensor on $S$. If $\I$ and $\II$ satisfy the Gauss-Codazzi equations, then there exists a spacelike immersion $\sigma:S \to\A^{2,1}$ having $\I$ and $\II$ as first and second fundamental form. Moreover if $\sigma$ and $\sigma'$ are two such immersions, then there exists a time-preserving isometry $\varphi$ of $\A^{2,1}$ such that $\sigma' = \varphi \circ \sigma$.
\end{theorem}

\section{Germs of spacelike immersions}
Let us now consider che case in which $S$ is an oriented surface, not necessarily simply connected. Let $\sigma: S \to(M,g)$ a spacelike immersion where $(M,g)$ is an oriented AdS spacetime.
As in the previous section, we can associate to $\sigma$ the pair $(\I,\II)$ of first and second  fundamental form, where $\II$ is  computed with respect to the unit normal vector $\nu$ of $\sigma$.
We will always assume that the orientation of $S$ and $\nu$ are compatible with the one on $M$.\\
We want to prove that the pair $(\I,\II)$ depends only on the \textit{germ} of $\sigma$, which is defined as follow:
\begin{definition}
    A \textit{germ} of a spacelike immersion of $S$ into an AdS spacetime is an equivalent class of spacelike immersions $\sigma:S \to(M,g)$ by the following relation: $\sigma:S \to(M,g)$ and $\sigma':S \to(M',g')$ are equivalent if there exist open subsets $U \supset \sigma(S)$ in $M$ and $U' \supset \sigma'(S)$ in $M'$ and a time-preserving isometry $f:(U,g)\to(U',g')$ such that $\sigma' = f \circ \sigma$.
\end{definition}
Given a pair $(\I,\II)$ on $S$ and $\pi: \widetilde{S}\to S$ a universal cover, one can take the pair $(\pi^*\I, \pi^*\II)$ on $\widetilde{S}$ that clearly satisfy the Gauss-Codazzi equations. By the existence part of Theorem \ref{thm:immersion of simply connected surface} there exists a spacelike immersion $\widetilde{\sigma}: \widetilde{S} \to\A^{2,1}$ having immersion data $(\pi^*\I, \pi^*\II)$. As a concequence of the uniqueness part of Theorem \ref{thm:immersion of simply connected surface} we have that associeted to $\widetilde{\sigma}$ there is a map $\rho:\pi_1(S) \to \text{Isom}_0(\A^{2,1})$ such that for every $\gamma \in \pi_1(S)$, $\widetilde{\sigma} \circ \gamma = \rho(\gamma) \circ \widetilde{\sigma}$. Moreover changing $\widetilde{\sigma}$ by post-composition with an isometry $f$ of $\A^{2,1}$ has the effect of conjugating $\rho$ by $f$.\\
Given $\sigma:S \to \A^{2,1}$ one can extend the immersion to an immersion of an open neighbourhood of $S \times\{0\}$ in $S\times\R$ into $\A^{2,1}$, by mapping $(x,t)$ to the point $\gamma(t)$ on the timelike geodesic $\gamma$ such that $\gamma(0)=\sigma(p)$ and $\gamma'(0)$ is the future normal vector of $\sigma$ at $x$.
\begin{lemma}\label{lem:tub metric}
    Given a spacelike immersion $\sigma:S\to \A^{2,1}$, the pull-back metric by means of the map $(p,t) \to exp_{\sigma(x)}(t\nu(x))$ has the expression:
    \begin{equation}\label{eq:tub metric}
        -dt^2 + \text{cos}^2(t)\I + 2\text{cos}(t)\text{sin}(t)\II + \text{sin}^2(t)\III,
    \end{equation}
    where $\I$, $\II$, $\III$ are the first, second and third fundamental form of $\sigma$ and third fundamental form is defined as $\III(\cdot,\cdot)=\I(B(\cdot),B(\cdot))$.
\end{lemma}
\begin{proof}
    \red{mettere proof?}
\end{proof}
Therefore, given a pair $(\I,\II)$, the expression \ref{eq:tub metric} provides a Lorentzian metric of constant curvature $-1$ on an open neighbourhood of $S \times\{0\}$ in $S\times\R$ into $\A^{2,1}$, and thus a germ of immersion of $S$ into and AdS spacetime with immersion data $(\I,\II)$. We conclude the above discussion with the following:
\red{da rivedere l'uso di $S$ e $\widetilde{S}$ sembra mal posto nel paper}
\begin{proposition}\label{thm:immersion data classification}
    Given a surface $S$, there are natural identifications between the following spaces:
    \begin{itemize}
        \item The space of pairs $(\I,\II)$ on $S$ which are solutions of the Gauss-Codazzi equations,
        \item The space of germs of spacelike immersions of $S$ into Anti-de Sitter manifolds,
        \item The space of spacelike immersions of $\widetilde{S}$ into $\A^{2,1}$, equivariant with respect to a representation $\rho:\pi_1(S) \to \text{Isom}_0(\A^{2,1})$, up to the action of $\text{Isom}_0\A^{2,1}$ by post-composition.
    \end{itemize}
\end{proposition}
When $S$ is a closed surface the equivariant immersion $\widetilde{\sigma}$ is necessarily an embedding, which can be extended to an embedding of $\widetilde{S}\times\R$ into a domain of dependence in $\A^{2,1}$. The representation $\rho:\pi_1(S) \to \PSL \times \PSL$ coincides with the holonomy of the MGH spacetime $(M,g) \cong \widetilde{S}\times\R$, after identifying $\pi_1(S)$ with $\pi_1(M)$, and therefore $\rho$ consists of a pair $(\rho_l,\rho_r)$ of positive Fuchsian representation.\\
Therefore, the embedding data $(\I,\II)$ permit to recover the pair of elements in $\mathcal{T}(S)\times\mathcal{T}(S)$ which parametrizes MGH Anti-de Sitter manifolds with compact Cauchy surfaces.

\section{Gauss map}
We can now define the Gauss map for spacelike surfaces in $\A^{2,1}$. Recall that the space of timelike geodesics in $\A^{2,1}$ is identified with $\H^2\times\H^2$, where the identification maps a geodesic of the form
\[
    L_{p,q}=\{ X \in\PSL \ | \ X \cdot q = p \}
\]
to the pair $(p,q) \in \H^2\times\H^2$. 
\begin{definition}
    Let $\sigma :S \to\A^{2,1}$ a spacelike immersion. The \textit{Gauss map} $G_\sigma : S \to \H^2 \times \H^2$ is defined as $G_\sigma(x) = (p,q)$ such that $L_{p,q}$ is the timelike orthogonal to $\text{Im}(d_x\sigma)$ at $\sigma(x)$.\\
    The components of the Gauss map, denoted by $\Pi_l, \Pi_r : S \to \A^{2,1}$, are called \textit{left} and \textit{right projections}. 
\end{definition}
The Gauss map $G_\sigma$ is natural with respect to the action of the isometry group, meaning that $G_{f\circ\sigma} = f \cdot G_\sigma$ for every $f \in\text{Isom}_0(\A^{2,1}) = \PSL\times\PSL$. The Gauss map is also invariant by reparametrization, in the sense that $G_{\sigma \circ \phi} = G_\sigma \circ \phi$ for a diffeomorphism $\phi:S'\to S$. Hence it makes sense to talk about the Gauss map of a spacelike surface in $\A^{2,1}$.

\begin{example}\label{ex:dual1}
    In Lemma \ref{lem:dual plane} we gave an isometric embedding of $\H^2$ in $\A^{2,1}$ with image the plane $P_{\text{Id}}$ dual to the identity. This isometric embedding was defined by sending $p\in\H^2$ to the unique order-two element in $\PSL$ fixing $p$, which by definition lies on the geodesic $L_{p,p}$. Moreover the geodesic $L_{p,p}$ is orthogonal to $P_{\text{Id}}$. Hence the Gauss map associated to this isometric embedding of $\H^2$ is just $p\mapsto (p,p).$
\end{example}
Geodesic leaving from $\1$ with velocity $\nu(p)$ meets $P_\1$ orthogonally at $\text{exp}((\pi/2)\nu(p))$, therefore, using Example \ref{ex:dual1}, one can recover the following:

\begin{lemma}\label{lem:Gaussid}
    Given a spacelike immersion $\sigma:S\to\A^{2,1}$, with future unit normal vector field $\nu$, if $\sigma(p)=\text{Id}$, then 
    \begin{equation}\label{eq:Gaussid}
        G_\sigma(p)=G_{P_\text{Id}}(\exp(\frac{\pi}{2}\nu(p))).
    \end{equation}
\end{lemma}
%\begin{proof}
%    \red{mettere proof}
    %It is a consequence of Example \ref{ex:dual1}. We need to observe that the geodesic leaving from Id with velocity $\nu(p)$ meets orthogonally $P_\text{Id}$ at $\exp((\pi/2)\nu(p)).$
%\end{proof}

Now we denote with $\hf\A^{2,1}$ the hyperboloid of future unit timelike vectors in $T_{\text{Id}}\A^{2,1}$ and consider the following map: 
\[
    \text{Fix}:T_{\text{Id}}^{1,+}\A^{2.1}\to\H^2
\]
defined so that $\text{Fix}(\nu)$ is the fixed point of the one-parameter elliptic group $\{\exp(t\nu)\;|\;t\in\R\}$. This map is equivariant for the action of $\PSL$, which acts on the hyperboloid $\hf\A^{2.1}$ by the adjoint representation and on $\H^2$ by the obvious action. Since both $\hf\A^{2.1}$ and $\H^2$ have constant curvature $-1$, it follows from equivariance that $\text{Fix}$ is an isometry.  \\
In terms of the map $\text{Fix},$ Equation \refeq{eq:Gaussid} reads as: 
\begin{equation}\label{eq:Gaussfix}
    G_\sigma(p)=(\text{Fix}(\nu(p)),\text{Fix}(\nu(p))),
\end{equation}
provided that $\sigma(p)=\text{Id}.$\\ Via Lemma \ref{lem:Gaussid} and the naturality, we can recover a different description of the Gauss map. Recalling the structure of Lie group of $\A^{2,1}\simeq\PSL$ we will denote by $R_{\gamma } $ and $L_{\gamma} $ the right and left multiplication by $\gamma\in\PSL$ respectively. 

\begin{lemma}
    Given a spacelike immersion $\sigma: S\to \A^{2.1}$ with future unit normal vector field $\nu$,
    \[
        G_\sigma(p)=(\text{Fix}((R_{\sigma(p)^{-1}})_* (\nu(p)),\text{Fix}((L_{\sigma(p)^{-1}})_* (\nu(p)))).
    \]
\end{lemma}

\begin{proof}
    If $\sigma(p)=\text{Id}$, then the equality holds true by Equation \refeq{eq:Gaussfix}. For the general case, the immersion $\sigma^{\prime} =(\text{Id},\sigma(p))\circ\sigma$ is such that $\sigma^{\prime}(p)=\text{Id}$, and the future normal vector at $\sigma^{\prime} (p)$ equals $\nu^{\prime} (p)=(R_{\sigma(p)^{-1}}))_{\ast} (\nu(p))$. Therefore: 
    \[
        G_{\sigma^{\prime} }(p)=(\text{Fix}((R_{\sigma(p)^{-1}})_{\ast} (\nu(p)),\text{Fix}((R_{\sigma(p)^{-1}})_{\ast} (\nu(p))))).
    \]
    Now by the naturality of the Gauss map it follows,     
    \begin{align*}
        G_\sigma(p)&=(\text{Id},\sigma(p)^{-1})\cdot G_{\sigma^{\prime}}(p) \\
        &=(\text{Fix}((R_{\sigma(p)^{-1}})_{\ast} (\nu(p)),\sigma(p)^{-1}\circ\text{Fix}((R_{\sigma(p)^{-1}})_{\ast} (\nu(p))))) \\
        &=(\text{Fix}((R_{\sigma(p)^{-1}})_{\ast} (\nu(p)),\text{Fix}((L_{\sigma(p)^{-1}})_{\ast} (\nu(p))))).
    \end{align*}
    where we used that Fix is equivariant with respect to the adjoint action on the hyperboloid $\hf\A^{2.1}$.
\end{proof}

We want now to prove formulae which express the pull-back of the hyperbolic metrics by the left and right projections. When applying these formulae to the embedding data of a surface in an MGH Cauchy compact Anti-de Sitter spacetime $(M,g)$, we obtain a pair of hyperbolic metrics whose isotopy classes are the parameters of $(M,g)$
in $\mathcal{T}(S)\times\mathcal{T}(S).$

\begin{proposition}\label{prop:left right pull-back metric}
    Let $\sigma:S\to\A^{2,1}$ be a spacelike immersion, let $\Pi_l,\Pi_r:S\to\H^2$ be the left and right projections and let $g_{\H^2}$ be the hyperbolic metric. Then
    \[
        \Pi_l^*g_{\H^2}=I((\text{id}-\mathcal{J}B)\cdot,(\text{id}-\mathcal{J}B)\cdot),
    \]
        and
    \[
        \Pi_r^*(g_{\H^2})=I((\text{id}+\mathcal{J}B)\cdot,(\text{id}+\mathcal{J}B)\cdot),
    \]
    where $I$ is the first fundamental form of $\sigma$, $\mathcal{J}$ its associated almost-complex structure, and $B$ the shape operator.
\end{proposition}

\begin{proof}
    \red{Mettere proof?}
\end{proof}
We collect here some consequences and remarks around Proposition \ref{prop:left right pull-back metric}.
\begin{itemize}[leftmargin=0.5cm]
\item Proposition \ref{prop:left right pull-back metric} shows that the differential of the left and right projections essentially has the expression 
\[
    d_x\sigma\circ(B\pm\JJ)~,
\]
up to post-composing with an isometry sending the image of $d_x\sigma$ to a fixed copy of $\H^2$. Since $B$ is $\I$-symmetric, $\JJ\circ B$ is traceless, and therefore 
\begin{equation}\label{eq:deteminant differential projections}
\det(B\pm\JJ)=1+\det B=-K_\I~.
\end{equation}
This shows that $\Pi_l$ is a local diffeomorphism at a point $x$ if and only if $\Pi_r$ is, which is the case if and only if the intrinsic curvature of $\I$ at $x$ is different from $0$.
\item Since the trace of $B\pm\JJ$ equals $2$, the differentials of $\Pi_l$ and $\Pi_r$ have either rank 2 or rank 1. (In fact, by \eqref{eq:deteminant differential projections}, when the differential of $\Pi_l$ has rank 1, the same holds for the differential of $\Pi_r$.) Hence the differential of the Gauss map $G:S\to\H^2\times\H^2$ is always non-singular.
\item If an immersed surface has the property that the curvature of the first fundamental form never vanishes, and if moreover $\Pi_l$ and $\Pi_r$ are globally injective, then the image of $G$ is the graph of a  diffeomorphism $F_\sigma$ between two subsets of $\H^2$,  called the \emph{associated map}. From Equation \eqref{eq:deteminant differential projections}, the Jacobians of $\Pi_l$ and $\Pi_r$ are equal, hence the associated map is area-preserving.
When $\Pi_l$ and $\Pi_r$ are only locally injective, but not globally, we still obtain an area-preserving local diffeomorphism $F_\sigma$ which is now defined between two hyperbolic surfaces, not globally isometric to subsets of $\H^2$.
\item More generally, as a consequence of the previous points, the image of $G$ is always a \emph{Lagrangian submanifold} in $\H^2\times\H^2$ with respect to the symplectic form 
\begin{equation} \label{eq symplectic form}
\Omega=\pi_l^*\omega_{\H^2}-\pi_r^*\omega_{\H^2}~,
\end{equation}
where $\omega_{\H^2}$ is the hyperbolic area form. 
This result has been proved in several works with different methods: see \cite{bonsante2017equivariant}, \cite{Seppi_2017}. Moreover the Lagrangian condition is \emph{locally} the only obstruction to inverting this construction, that is, to realizing an immersed surfaces in $\H^2\times\H^2$ locally as the image of the Gauss map of a spacelike immersion in $\A^{2,1}$.
\item Given a spacelike immersion $\sigma$,  the normal evolution of $\sigma$ is defined as 
\[
    \sigma_t(x)=\exp_{\sigma(x)}(t\nu(x))~,
\]
where $\nu$ is the future unit normal vector field. In general $\sigma_t$ may fail to be an immersion for $|t|$ large. When it is an immersion, the computation of the metric in Lemma \ref{lem:tub metric} shows that the image of $\sigma_t$ at $x$ is orthogonal to the geodesic $\gamma(t)=\exp_{\sigma(x)}(t\nu(x))$. In other words, the Gauss map of $\sigma_t$ is equal to the Gauss map of $\sigma$. Hence with respect to the previous point, given a spacelike immersion which have the same  Lagrangian submanifold of $\H^2\times\H^2$ as Gauss map image.
\end{itemize}