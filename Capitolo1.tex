\chapter{Preliminaries on Lorentzian geometry}

The scope of this introductory chapter is to briefly recall basic facts about Lorentzian manifolds as later on we will work on Lorentzian manifolds with constant sectional curvature $K=-1$.
We will show that, as in Riemannian geometry, Lorentzian manifolds of constant sectional curvature are locally isometric. We will focus on Lorentzian manifolds with maximal isometry group and we will show that generally there are manifolds with maximal isometry group that are not simply connected.
\section{Basic definitions}
A \textit{Lorentzian metric} on a manifold of dimension $n+1$ is a non degenerate symmetric 2-tensor $g$ of signature $(n,1)$.
A \textit{Lorentzian manifold} is a connected manifold $M$ equipped with a Lorentzian metric $g$.\\
We say that a vector $0\neq v \in TM$ is \textit{spacelike, lightlike, timelike} if $g(v,v)$ is respectively positive, zero or negative.
More generally we say that a linear subspace $V \in T_x M$ is \textit{spacelike, lightlike, timelike} if the restriction of $g_x$ to $V$ is positive definite, degenerate or indefinite, respectively.\\
The set of lightlike vectors, together with the null vector, disconnects the tangent space $T_x M$ into three regions: two convex open cones formed by timelike vectors and the region of spacelike vectors.
As a consequence the set of timelike vectors in $TM$ is either connected or has two connected components.
In the latter case $M$ is said to be \textit{time-orientable}, and a \textit{time orientation} is a choice of one of these components.
Vectors in the chosen component are said to be \textit{future-directed}, and vectors in the other component are said to be \textit{past-directed}.
In analogy with tangent vectors, a differetiable curve is \textit{spacelike, lightlike, timelike} if the tangent vector is respectively spacelike, lightlike or timelike at every point. Moreover, we say that a curve is \textit{causal} if the tangent vector is timelike or lightlike at every point.  Given a point $x$ in a time-oriented Lorentzian manifold, we also define the \textit{future} of $x$ as the set $I^+(x)$ of points connected to $x$ by a future-directed causal curve. In an analogous way we define the \textit{past} $I^-(x)$ of $x$.\\
As in the Riemannian setting, one can define on a Lorentzian manifold $(M,g)$ the \textit{Levi-Civita connection}. Following the analogy with Riemannian geometry, the Levi-Civita connection determines the Riemann curvature tensor defined by: 
\[
    R(u,v)w=\nabla_u\nabla_v w-\nabla_v\nabla_u w-\nabla_{[u,v]}w.
\]  

\noindent We then say that a Lorentzian manifold has constant sectional curvature $K$ if
\begin{equation}\label{sectionalcurvature}
    g(R(u,v)v,u)=K(g(u,u)g(v,v)-g(u,v)^2) 
\end{equation}
for every pair of vectors $u,v \in T_{x}M$ and every $x\in M$. We recall that in the Lorentzian setting the sectional curvature can be defined only for planes in $T_{x}M$ where $g$ is non-degenerate.\\

\section{Maximal isometry group and geodesic completeness}
Lorentzian manifolds of constant curvature $K$ are locally isometric, more precisely
\begin{lemma}
    Let $M$ and $N$ be Lorentzian manifolds of constant curvature $K$. Every linear isometry $L: T_x M \to T_y N$ extends to an isometry $f: U \to V$ where $U$ and $V$ are neighborhoods of $x$ and $y$ respectively. Any two extensions $f:U\to V$ and $f' : U' \to V'$ coincide on $U \cap U'$. Moreover, if $M$ is simply connected and $N$ is geodesically complete $L$ extends to a local isometry $f:M\to N$.
\end{lemma}
As in the Riemannian case the proof is a consequence of the classical Cartan-Ambrose-Hicks Theorem (for a reference see \cite{piccione2005single}).\\
As a consequence of the lemma
\begin{corollary}
    Let $M$ and $N$ be simply connected, geodesically complete Lorentzian manifolds of constant curvature $K$. Then any linear isometry $L: T_x M \to T_y N$ extends to a global isometry $f: M \to N$. In particular there is a unique simply connected, geodesically complete Lorentzian manifold of constant curvature $K$ up to isometries.
\end{corollary}
\noindent In Chapter \ref{chapter:2} and \ref{chapter:3} we will construct explicit models for $K=-1$.\\
Another consequence is that the group of isometries $\text{Isom}(M)$ can be realized as a subset of $\text{ISO}(T_{x_o}, TM)$, where $x_0 \in M$ and $\text{ISO}(T_{x_o}, TM)$ is the fiber bundle over $M$ whose fiber over $x$ is the space of linear isometries of $T_{x_0}M$ into $T_xM$.
It follows that the maximal dimension of $\text{Isom}(M)$ is $\text{dim}(O(n+1)) + n +1 = (n+1)(n+2)/2$.\\
\begin{definition}
    A Lorentzian manifold M has \textit{maximal isometry group} if the action of Isom(M) is transitive and for every $x\in M$ every linear isometry $L: T_xM \to T_xM$ extends to an isometry of M.\\
    Equivalently $M$ has maximal isometry group if $\text{Isom}(M) \subset \text{ISO}(T_xM, TM)$ is a bijection.
\end{definition}
It follows that if $M$ has maximal isometry group the dimension of the isometry group is maximal.\\
From the corollary it follows that every simply connected Lorentzian manifold $M$ has maximal isometry group if it is geodesically complete and has constant curvature. The converse holds:
\begin{lemma}\label{lem:constant curvature}
    If $M$ is a Lorentzian manifold with maximal isometry group then $M$ has constant sectional curvature and is geodesically complete.
\end{lemma}

\begin{proof}
    Fix a point $x\in M$, the identity component of $\text{O}(T_x M)\simeq \text{O}(n,1)$ acts transitively on spacelike planes, therefore there is a constant $K$ such that Equation \ref{sectionalcurvature} holds for every pair $(u,v)$ of vectors tangent at $x$ which generate a spacelike plane. Now, for every point $x\in M$ both sides of Equation \ref{sectionalcurvature} are polynomial in $u,v \in T_xM$. Since the set of pairs $(u,v)$ which generate spacelike planes is open in $T_{x}M\times T_{x}M$, Equation \ref{sectionalcurvature} must hold for every pair of vectors $u,v \in T_xM$. Since Isom($M$) acts transitively on $M$, it follows that the sectional curvature of $M$ is constant.\\
    Let us now show that the manifold is geodesically complete. Suppose $\gamma$ is a parametrized geodesic with $\gamma(0)=x$ and $\gamma^{\prime} (0)=v\in T_xM,$ which is defined for a finite time $T>0.$ Let $T_0=T-\epsilon>0.$ By assumption one can find an isometry $f:M\to M$ such that $f(x)=\gamma(T_0)$ and $df_x(v)=\gamma^{\prime}(T_0).$ Then $t\to f\circ\gamma(t-T_0)$ is a parametrized geodesic which provides a continuation of $\gamma$ beyond $T$.
\end{proof}


For simply connected Lorentzian manifold there is the following classification result:
\begin{proposition}
    Let $M_K$ be a simply connected Lorentzian manifold of constant curvature $K$ with maximal isometry group. If $M$ is a Lorentzian manifold of constant curvature $K$ then:
    \begin{itemize}
        \item M is geodesically complete if and only if there is a local isometry $p:M_K \to M$ which is a universal covering.
        \item M has maximal isometry group if and only if $\text{Aut}(p:M_K \to M)$ is normal in $\text{Isom}(M_K)$
    \end{itemize}
\end{proposition} 

\begin{proof}
    Suppose $M$ is geodesically complete, then by lifting the metric to the universal cover $\widetilde{M}$ one gets a simply connected geodesically complete Lorentzian manifold of constant sectional curvature $K$ which is isometric to $M_K$. The covering map $p:M_K\to M$ is then a local isometry by construction. The converse is straightforward. \\   
    Now let $\Gamma$ be $\text{Aut}(p:M_K\to M),$ which is a discrete subgroup of Isom($M_K$). Thus $M$ is obtained as the quotient $M=M_{K}/\Gamma$, where $\Gamma$ acts freely and properly discontinuously on $M_K$. The isometry group of $M$ is isomorphic to $N(\Gamma)/\Gamma$, where by $N(\Gamma)$ we denote the normalizer of $\Gamma$ in Isom($M_K$). The isomorphism is based on the observation that any isometry of $M_K$ which normalizes $\Gamma$ descends to an isometry of $M$, and conversely the lifting of any isometry of $M$ must be in $N(\Gamma).$\\
    Hence the condition that $M$ has maximal isometry group is equivalent to the condition that every element $f$ of $\text{Isom}(M_K)$ descends to the quotient to an isometry of $M$. This is in turn equivalent to the condition that $f\Gamma f^{-1}=\Gamma$ for every $f\in \text{Isom}(M_K)$, which is the same as saying that $\Gamma$ is normal in $\text{Isom}(M_K)$.
\end{proof}
It follows that any isometry between connected open subsets of a Lorentzian manifold $M$ with maximal isometry group extends to a global isometry. 
In particular, if $M_K$ is a Lorentzian manifold of constant sectional curvature $K$ with maximal isometry group, than any Lorentzian manifold $M$ of constant sectional curvature $K$ admits a natural $(\text{Isom}(M_K),M_K)$-structure whose charts are isometries between open subsets of $M$ and open subsets of $M_K$.
We will refer to Lorentzian manifolds of constant sectional curvature $K$ with maximal isometry group as models of constant sectional curvature $K$. In the following chapters we will study several models of constant sectional curvature $-1$ called models of Anti-de Sitter geometry.