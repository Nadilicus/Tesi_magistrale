\chapter{Anti-de Sitter space in dimension (2+1)}

\section{The $\PSL$-model}
In dimension three there is a special model which endows Anti-de Sitter space with a Lie group structure.
Consider the vector space $\mathcal{M}(2,\R)$ of $2 \times 2$ matrices with real entries and the quadratic form $q= -det$ that has signature $(2,2)$.
This gives an identification between $(\mathcal{M}(2,\R), -det)$ and $(\R^{2,2},q_{2,2})$, unique up to composition by elements in $O(2,2)$, and under this isomorphism $\H^{2,1}$ is identified with the Lie group $SL(2,\R)$.\\
$SL(2,\R)\times SL(2,\R)$ acts linearly on $\mathcal{M}(2,\R)$ by left and right multiplication:
\[
    (A,B) \cdot X := AXB^{-1}    
\]
preserving the quadratic form $q = -det$. Therefore it induces a representation
\[
    SL(2,\R)\times SL(2,\R) \to O(\mathcal{M}(2,\R),q).
\]
The kernel of this representation is given by $K=\{ (\1,\1),(-\1,-\1) \}$, and by dimensional argument it is the connected component of the identity:
\[
    \text{Isom}_0(\H^{2,1}) \cong SO_0(\mathcal{M}(2,\R),q) \cong SL(2,\R)\times SL(2,\R) / K.
\]
Using this modle there is a natural identification of $\A^{2,1}$ with the Lie group $\PSL$ and
\[
    \text{Isom}_0(\A^{2,1}) \cong \PSL \times \PSL
\]
acting by left and right multiplication on $\PSL$.
The stabilizer of the identity in $\text{Isom}_0(\A^{2,1})$ is the diagonal subgroup $\Delta < \PSL \times \PSL$\\

\section{The boundary of $\PSL$}
From the identification between $\A^{2,1}$ and $\PSL$ we obtain an identification of $\partial \A^{2,1}$ with the boundary of $\PSL$ into $P(\mathcal{M}(2,\R))$, which is the projectivization of rank 1 matrices:
\[
    \partial \A^{2,1} = \{ [X] \in P(\mathcal{M}(2,\R)) \ | \ rank(X)=1 \}.
\]
We have a homeomorphism
\begin{align}
    \partial \A^{2,1} \to \T \\
    [X] \to (\text{Im}X, \text{Ker}X)
\end{align}
equivariant under the action on $\A^{2,1}$ of $\PSL \times \PSL$.\\

\begin{lemma}
The inversion map $\iota [X] = [X]^{-1}$ is a time-reversing isometry of $\A^{2,1}$ which induces the homeomorphism $(x,y) \to (y,x)$ on $\partial \A^{2,1} \cong \T$.
\end{lemma}
\begin{proof}
The map $\iota$ is equivariant with respect to the isomorphism of $\PSL \times \PSL$ that switches the two factors.
To show that it is an time-reversing isometry it thus suffices to check the differential at the identity, which is clearly $d_\1 \iota = -\1$.\\
To check the second claim we observe that for an invertible $2 \times 2$ matrix we have $(detX)X^{-1} = (trX)\1 - X$ by Cayley-Hamilton theorem, so that projectively $[X^{-1}]=[(trX) \1 - X]$ along the boundary.
This shows that the inversion map on $\A^{2,1}$ extends to the transormation $[X] \to [(trX)\1 - X]$ along the boundary.
Let $X$ be a rank 1 matrix, it is traceless if and only if $X^2 =0$, that is, if and only if $\text{Ker}X=\text{Im}X$, and in this case the statement is easily proven.
If $trX \neq 0$ then $X$ is diagonalizable with eigenvalues $0$ and $trX$. Moreover $\text{Ker}X$ and $\text{Im}X$ are the corresponding eigenspaces and it is easily seen that $\text{Ker}((trX)\1 -X) = \text{Im}X$ and $\text{Im}((trX)\1 -X) = \text{Ker}X$.
\end{proof}