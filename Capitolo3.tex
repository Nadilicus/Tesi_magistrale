\chapter{Anti-de Sitter space in dimension (2+1)} \label{chapter:3}

\red{Numerare equazioni, aggiungiere riferimenti}
\section{The $\PSL$-model}
In dimension three there is a special model which endows Anti-de Sitter space with a Lie group structure.
Consider the vector space $\mathcal{M}(2,\R)$ of $2 \times 2$ matrices with real entries and the quadratic form $q= -det$ that has signature $(2,2)$.
This gives an identification between $(\mathcal{M}(2,\R), -det)$ and $(\R^{2,2},q_{2,2})$, unique up to composition by elements in $O(2,2)$, and under this isomorphism $\H^{2,1}$ is identified with the Lie group $SL(2,\R)$.\\
$SL(2,\R)\times SL(2,\R)$ acts linearly on $\mathcal{M}(2,\R)$ by left and right multiplication:
\[
    (A,B) \cdot X := AXB^{-1}    
\]
preserving the quadratic form $q = -det$. Therefore it induces a representation
\[
    SL(2,\R)\times SL(2,\R) \to O(\mathcal{M}(2,\R),q).
\]
The kernel of this representation is given by $K=\{ (\1,\1),(-\1,-\1) \}$, and by dimensional argument it is the connected component of the identity:
\[
    \text{Isom}_0(\H^{2,1}) \cong SO_0(\mathcal{M}(2,\R),q) \cong SL(2,\R)\times SL(2,\R) / K.
\]
Using this model there is a natural identification of $\A^{2,1}$ with the Lie group $\PSL$ and
\[
    \text{Isom}_0(\A^{2,1}) \cong \PSL \times \PSL
\]
acting by left and right multiplication on $\PSL$.
The stabilizer of the identity in $\text{Isom}_0(\A^{2,1})$ is the diagonal subgroup $\Delta < \PSL \times \PSL$.\\

\section{The boundary of $\PSL$}
From the identification between $\A^{2,1}$ and $\PSL$ we obtain an identification of $\partial \A^{2,1}$ with the boundary of $\PSL$ into $P(\mathcal{M}(2,\R))$, which is the projectivization of rank 1 matrices:
\[
    \partial \A^{2,1} = \{ [X] \in P(\mathcal{M}(2,\R)) \ | \ rank(X)=1 \}.
\]
We have a homeomorphism
\begin{align}
    \partial \A^{2,1} \to \T \\
    [X] \to (\text{Im}X, \text{Ker}X)
\end{align}
equivariant under the action of $\PSL \times \PSL$, acting on $\partial\A^{2,1}$ as the natural extension of the action on $\A^{2,1}$, and by left multiplication on $\T$.\\
The equivariance is easily checked observing that $\text{Im}(AXB^{-1}) = A \cdot \text{Im}(X)$ and $\text{Ker}(AXB^{-1}) = B \cdot \text{Ker}(X)$.
\begin{lemma}
    The inversion map $\iota [X] = [X]^{-1}$ is a time-reversing isometry of $\A^{2,1}$ which induces the homeomorphism $(x,y) \to (y,x)$ on $\partial \A^{2,1} \cong \T$.
\end{lemma}
\begin{proof}
    The map $\iota$ is equivariant with respect to the isomorphism of $\PSL \times \PSL$ that switches the two factors.
    To show that it is an time-reversing isometry it thus suffices to check the differential at the identity, which is clearly $d_\1 \iota = -\1$.\\
    To check the second claim we observe that for an invertible $2 \times 2$ matrix we have $(detX)X^{-1} = (trX)\1 - X$ by Cayley-Hamilton theorem, so that projectively $[X^{-1}]=[(trX) \1 - X]$ along the boundary.
    This shows that the inversion map on $\A^{2,1}$ extends to the transormation $[X] \to [(trX)\1 - X]$ along the boundary.
    Let $X$ be a rank 1 matrix, it is traceless if and only if $X^2 =0$, that is, if and only if $\text{Ker}X=\text{Im}X$, and in this case the statement is easily proven.
    If $trX \neq 0$ then $X$ is diagonalizable with eigenvalues $0$ and $trX$. Moreover $\text{Ker}X$ and $\text{Im}X$ are the corresponding eigenspaces and it is easily seen that $\text{Ker}((trX)\1 -X) = \text{Im}X$ and $\text{Im}((trX)\1 -X) = \text{Ker}X$.
\end{proof}

Using the half-plane model for the hyperbolic space $\H^2$, $\S$ is identified to the boundaty at infinity $\partial \H^2$ and $\PSL$ corresponds to $Isom_0(\H^2)$, which acts on $\S$ in the canonical way.
Therefore one can identify $\partial \A^{2,1}$ with $\partial \H^2 \times \partial \H^2$ and we can interpret the convergence to $\partial \A^{2,1}$ in the following way:
\begin{lemma}
    A sequence $[X_n] \in \A^{2,1}$ converges to $(x,y) \in \partial \A^{2,1} \cong \T$ if and only if for every $p\in\H^2$, $X_n(p) \to x$ and $X_n^{-1}(p) \to y$.
\end{lemma}

\begin{proof}
    \red{no proof per ora}
\end{proof}

\red{vedere se aggiungere cose sul left e right ruling}

\section{Geodesics in $\PSL$}
Let us start by understanding the geodesics through the identity. Using the Lie group structure of $\A^{2,1}$ it suffices to understand the one-parameter subgroup of $\A^{2,1}$
(the necessary tools in Lie groups theory are introduced in \cite{bonsanteseppi}). We get the following:
\begin{itemize}
    \item Timelike geodesics are, up to conjugacy, of the form
    \[ \begin{pmatrix}
        \text{cos}(t) & -\text{sin}(t) \\
        \text{sin}(t) & \text{cos}(t)
    \end{pmatrix} \]
    under the identification of $\PSL$ with $Isom(\H^2)$ they are elliptic one-parameter groups fixing a point in $\H^2$.
    \item Spacelike geodesics are, up to conjugacy
    \[ \begin{pmatrix}
        \text{cosh}(t) & \text{sinh}(t) \\
        \text{sinh}(t) & \text{cosh}(t)
    \end{pmatrix} \]
    these are hyperbolic one-parameter groups fixing two points in the boundary of $\H^2$
    \item Lightlike geodesics are the parabolic one-parameter groups conjugate to
    \[ \begin{pmatrix}
        1 & t \\
        0 & 1
    \end{pmatrix} \]
\end{itemize}
Using this description of timelike geodesics through $\1$, we can also interpret the duality in $\A^{2,1}$. Recalling that the dual plane of a point $A$ is the set of antipodal points along timelike geodesics through $A$, one sees that the dual plane of $\1$ consists of elliptic order-two isometries of $\H^2$. Equivalently, this is the set of (projective classes) of traceless matrices, that is (by the Cayley-Hamilton theorem)
\[
    P_\1 = \{ [J]\in\PSL \ | \ J^2 = -\1 \}.
\]
The boundary at infinity of $P_\1$ is made of traceless matrices of rank 1.\\
The stabilizer of $\1$ is the diagonal subgroup of $\PSL \times \PSL$, and it aso acts on the dual plane $P_\1$ by conjugation. Therefore the following statement is straightforward:
\begin{lemma}\label{lem:dual plane}
    The map from $\H^2$ to $P_\1$, sending $p\in\H^2$ to the elliptic order-two element in $\PSL$ fixing $p$, is a $\PSL$-equivariant isometry.
\end{lemma}

\noindent\textit{Timelike geodesics}.
To get a complete description of timelike geodesics it suffices to let the isometry group of $\A^{2,1}$ act on geodesics through the identity.
\begin{proposition}
    There is a homeomorphism between the space of timelike geodesics of $\A^{2,1}$ and $\H^2 \times \H^2$.
    The homeomorphism is equivariant for the action of $Isom_0(\A^{2,1}) \cong \PSL \times \PSL$.
\end{proposition}
\begin{proof}
    \red{mettere proof}
\end{proof}

\noindent\textit{Spacelike geodesics}.
Let $l$ be a geodesic of $\H^2$. The one-parameter group of hyperbolic transormation fixing $l$ as an oriented geodesic is a spacelike geodesic through the identity.
\begin{proposition}
    There is a homeomorphism between the space of oriented spacelike geodesics of $\A^{2,1}$ and the product of two copies of $\partial \H^2 \times \partial \H^2 \ \setminus \ \Delta$. 
    The homeomorphism is equivariant for the action of $Isom_0(\A^{2,1}) \cong \PSL \times \PSL$.
\end{proposition}
\begin{proof}
    \red{mettere proof}
\end{proof}