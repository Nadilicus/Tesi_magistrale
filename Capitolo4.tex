\chapter{Globally hyperbolic spacetime}\label{chapter:4}

\red{Numerare equazioni, aggiungiere riferimenti}\\
\red{nelle prime sezioni vanno aggiunte ancora cose}\\
\red{aggiungiere le def in $\A^{2,1}$ e dire che è uguale a farlo in $\AS^2,1$}\\
From now on we will only work with Lorentzian manifold of dimension $(2+1)$.

%Def di acausal set in \AS e \A (magari diretto come grafico)
%Lemma di caratt. degli acausal set come grafico
% Def di spacelike surface
% Spacelike surfaces sono acausal
%def di invisible domain 
%def di achronal meridian
\section{Achronal and acausal set}
\red{veloce intro, magari mettere alcuni risultati con o senza dim}

\begin{definition}
    A subset $X \subset \AS^{2,1} \cup \partial\AS^{2,1}$ is \textit{achronal} (resp. \textit{acausal}) is no pair of points in $X$ are connected by timelike (resp. causal) lines in $\AS^{2,1}$
\end{definition}
Consider the Poincaré model $\D\times\R$ of $\AS^{2,1}$ with the metric $g_{\mathbb{S}^2} - dt^2$. The following lemma gives a charecterization of achronal/acausal set:
\begin{lemma}
    A subset of $\AS^{2,1} \cup \partial\AS^{2,1}$ is achronal (resp. acausal) if and only if it is the graph of a function $\text{f} : D \to \R$ that is 1-Lipschitz (resp. strictly 1-Lipschitz) with respect to the distance induced by the hemispherical metric $g_{\mathbb{S}^2}$, where $D$ denotes the projection of $X$ to the $\D$ factor.
\end{lemma} 
\begin{proof}
    \red{mettere proof}
\end{proof}

\begin{definition}
    Given a surface $S$ and a Lorentzian manifold $(M,g)$, a $C^1$ immersion $\sigma : S \to M$ is \textit{spacelike} if the pull-back metric $\sigma^* g$ is a Riemannian metric. In this setting $\sigma(S)$ is said to be a \textit{spacelike surface}.
\end{definition}
A spacelike surface is locally acausal, if the immersion is proper we have the following global result:
\begin{lemma}
    Any properly embedded surface in $\AS^{2,1}$ is acausal.
\end{lemma}
\begin{proof}
    \red{mettere proof}
\end{proof}

\begin{definition}
    Let $X$ be an achronal set in $\AS^{2,1} \cup \partial\AS^{2,1}$, the \textit{invisible domain} of $X$ is the subset $\Omega(X) \subset \AS^{2,1}$ of points which are connected to $X$ by no causal path.
\end{definition}
\begin{definition}
    An \textit{achronal meridian} is a subset $\Lambda$ of $\partial\AS^{2,1}$ that os the graph of a 1-Lipschitz function $f: \partial\D\to\R$.
\end{definition}
The iportance of these definitions will be evident in the following sections.
The invisible domain of a achronal meridian will be fundamental tool in the study of hyperbolic spacetimes.
 

%def di domain of dependence e globally hyperbolic spacetime
%prop 4.4.6
%prop 4.5.5
%vedere se aggiungere cose sulla convessità della sezione 4.6

%studiamo MGH al variare del genere della superficie di Cauchy. Ci interessiamo solo di genere >1. Se genere =0 non ci sono globally hyperbolic spacetime.
%Prop 5.1.3, Def 5.1.4 MGH, Cor 5.1.5
\section{Globally hyperbolic spacetime}
\red{Aggiungere veloce intro forse}
\begin{definition}
    Given an achronal subset $X$ in a Lorentzian manifold $(M,g)$, the \textit{domain of dependance} of $X$ is the set
    \[
        \mathcal{D}(X)= \{ p \in M \ | \ \text{every inextensible causal curve through p meet X} \}.
    \]
    If $\mathcal{D}(X)=M$ we say that M is  \textit{globally hyperbolic} whit \textit{Cauchy surface} $X$.
\end{definition}
Globally hyperbolic spacetime have strong geometric property we summarize in the following theorem:
\begin{theorem}
    Let $M$ be a globally hyperbolic spacetime, then
    \begin{itemize}
        \item Any two Cauchy surfaces are diffeomorphic.
        \item There  exists a submersion $\tau : M \to \R$ whose fibers are Cauchy surfaces.
        \item $M$ is diffeomorphic to $\Sigma \times \R$ where $\Sigma$ is a Cauchy surface.
    \end{itemize}
\end{theorem}
The aim of this section is to study maximal globally hyperbolic (MGH) Anti-de Sitter spacetimes containing a compact Cauchy surface of genus $n$ (in short we say that the globally hyperbolic spacetime has genus $n$). We will be interested mainly in the case $n\geq 2$ cause later on we will study MGH spacetime whose Cauchy surface is a closed hyperbolic surface.\\ 
\begin{observation}
    It can be shown that spacelike surfaces in $\A^{2,1}$ are properly embedded and diffeomorphic to $\R^2$.
    In particular there are no globally hyperbolic Anti-de Sitter spacetime of genus zero (\cite{bonsanteseppi}).
\end{observation}
\red{magari aggiungere effetivamente questo pezzo}
\begin{proposition}\label{prop:GH_geometry}
    Let $M$ be a globally hyperbolic Anti-de Sitter spacetime of genus $n\geq 1$. Then
    \begin{itemize}
        \item The developing map $dev: \widetilde{M} \to\A^{2,1}$ is injective.
        \item If $\Sigma$ is a Cauchy surface of $M$, then the image of dev is contained in $\Omega(\Lambda)$ where $\Lambda$ is the boundary of $dev(\widetilde{\Sigma})$.
        \item If $\rho : \pi_1(M) \to \text{Isom}(\A^{2,1})$ is the holonomy representetion, $\rho(\pi_1(M))$ acts freely and properly discontinuously on $\Omega(\Lambda)$ and $\Omega(\Lambda) / \rho(\pi_1(M))$ is a globally hyperbolic spacetime containig $M$.
    \end{itemize}
\end{proposition}
\begin{proof}
    \red{mettere proof}
\end{proof}
\begin{definition}
    A globally hyperbolic Anti-de Sitter spacetime $(M,g)$ is said to be \textit{maximal} if any isometric embedding of $(M,g)$ into a globally hyperbolic Anti-de Sitter spacetime $(M',g')$ which sends a Cauchy surface of $(M,g)$ to a Cauchy surface of $(M'.g')$ is surjective.
\end{definition}
Following from Proposition \ref{prop:GH_geometry} we have:
\begin{corollary}
    A globally hyperbolic Anti-de Sitter spacetime $M$ is maximal if and only if $\widetilde{M}$ is isometric to the invisible domain of a proper achronal meridian in $\A^{2,1}$.
\end{corollary}

%sezione 5.4 e 5.5 su MGH di genere >1
\section{Genus $n\geq 2$}
Let $\Sigma_n$ be an oriented surface of genus $n\geq 2$.
\begin{definition}
    A representetion $\rho: \pi_1(\Sigma_n) \to \PSL$ is \textit{positive Fuchsian} if there is a $\rho$-equivariant orientation-preserving homeomorphism $\delta : \widetilde{\Sigma_n}\to\H^2$.
\end{definition}
The following classical result in Teich\"uller theory is essential for the construction of MGH spacetime of genus $n\geq 2$.
\begin{lemma}
    Given two positive Fuchsian representetion $\rho_l, \rho_r : \pi_1(\Sigma_n) \to \PSL$, any $(\rho_l, \rho_r)$-equivariant orientation-preserving homeomorphism of $\H^2$ extends continuously to an orientation-preserving homeomorphism of $\H^2\cup\S$. Moreover, its extension $\varphi : \S\to\S$ is the unique $(\rho_l, \rho_r)$-equivariant orientation-preserving homeomorphism of $\S$.
\end{lemma}
By $(\rho_l, \rho_r)$-equivariance of $\varphi$ we mean that for every $\gamma \in \pi_1(\Sigma_n)$:
\[
    \varphi \circ \rho_l(\gamma) = \rho_r(\gamma)\circ\varphi.
\]
Given two positive Fuchsian representetion $\rho_l, \rho_r : \pi_1(\Sigma_n) \to \PSL$ we will consider the associated representetion in Anti-de Sitter geometry given by
\[
    \rho = (\rho_l, \rho_r) : \pi_1(\Sigma_n) \to \text{Isom}_0(\A^{2,1}) \cong \PSL \times \PSL.
\]
In this setting we define $\Lambda(\rho)$ as the graph of $\varphi: \S\to\S$ defined by $\rho$, and $\Omega_\rho := \Omega(\Lambda(\rho))$ its invisible domain in $\A^{2,1}$.
\begin{proposition}
    The domain $\Omega_\rho$ is invariant under the isometric action of $\pi(\Sigma_n)$ on $\A^{2,1}$ induced by $\rho$. Moreover $\pi_1(\Sigma_n)$ acts freely and properly discontinuously on $\Omega_\rho$ and the quotient is a MGH spacetime of genus $n$ and holonomy $\rho$.
\end{proposition}
\begin{proof}
    \red{metter proof, serve prop 4.5.4}
\end{proof}
$\Lambda(\rho)$ is also the unique achronal meridian in $\partial\A^{2,1}$ invariant under the action of $\pi_1(\Sigma_n)$ induced by $\rho$ making $M_\rho = \Omega_\rho / \rho(\pi_1(\Sigma_n))$ the unique spacetime with holonomy $\rho$.\\
One can also prove that if $M$ is a time-oriented MGH spacetime it is of the form $M_\rho$. More precisely:
\begin{proposition}
    Let $M$ be an oriented, time-oriented, globally hyperbolic spacetime of genus $n\geq 2$ and let us endow a Cauchy surface $\Sigma$ with the orientation induced by the future normal vector. Then the left and right components of the holonomy $\rho=(\rho_l,\rho_r): \pi_1(\Sigma)\to\PSL\times\PSL$ are positive Fuchsian representetion.
\end{proposition}


We conclude by stating the classification result. Let the \emph{deformation space} of MGH spacetimes of genus $n$ be:
$$\mathcal{MGH}(\Sigma_n)=\{g\text{ MGH AdS metric on }\Sigma_r\times\R\}/\mathrm{Diff}_0(\Sigma_n\times\R)~,$$
where the group of diffeomorphisms isotopic to the identity acts by pull-back. The holonomy map takes value in the space of representations of $\pi_1(\Sigma_r)$ into $\PSL\times\PSL$ and is well-defined on the quotient $\mathcal{MGH}(\Sigma_r)$.
We proved that the left and right components of the holonomy of elements of $\mathcal{MGH}(\Sigma_n)$ are positive Fuchsian representations, and the space of these representations up to conjugacy is identified with the Teichm\"uller space of $\Sigma_n$
\[
    \mathcal{T}(\Sigma_n)\cong\{\rho:\pi_1(\Sigma_n)\to\PSL\text{ positive Fuchsian representations}\}/\PSL~.
\]
Therefore the holonomy map can be considered as a map 
from $\mathcal{MGH}(\Sigma_n)$ with values in $\mathcal{T}(\Sigma_n)\times\mathcal{T}(\Sigma_n)$.
Hence, we can summarize this section with the following theorem of Mess.

\begin{theorem} \label{thm:classification rgeq2}
The holonomy map $$\rho:\mathcal{MGH}(\Sigma_n)\to\mathcal{T}(\Sigma_n)\times\mathcal{T}(\Sigma_n)$$ is a homeomorphism.
\end{theorem}