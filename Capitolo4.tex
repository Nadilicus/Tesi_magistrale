\chapter{Globally hyperbolic spacetime}\label{chapter:4}
From now on we will only work with Lorentzian manifold of dimension $(2+1)$.

\section{Achronal and acausal set}
\red{veloce intro}

\begin{definition}
    A subset $X \subset \AS^{2,1} \cup \partial\AS^{2,1}$ is \textit{achronal} (resp. \textit{acausal}) is no pair of points in $X$ are connected by timelike (resp. causal) lines in $\AS^{2,1}$
\end{definition}
Consider the Poincaré model $\D\times\R$ of $\AS^{2,1}$ with the metric $g_{\mathbb{S}^2} - dt^2$. The following lemma gives a charecterization of achronal/acausal set:
\begin{lemma}
    A subset of $\AS^{2,1} \cup \partial\AS^{2,1}$ is achronal (resp. acausal) if and only if it is the graph of a function $\text{f} : D \to \R$ that is 1-Lipschitz (resp. strictly 1-Lipschitz) with respect to the distance induced by the hemispherical metric $g_{\mathbb{S}^2}$, where $D$ denotes the projection of $X$ to the $\D$ factor.
\end{lemma} 
\begin{proof}
    \red{mettere proof}
\end{proof}

The following definitions will we fundamental in the study of hyperbolic spacetimes.
\begin{definition}
    Let $X$ be an achronal domain in $\AS^{2,1} \cup \partial\AS^{2,1}$, the \textit{invisible domain} of $X$ is the subset $\Omega(X) \subset \AS^{2,1}$ of points which are connected to $X$ by no causal path.
\end{definition}
\begin{definition}
    An \textit{achronal meridian} is a subset $\Lambda$ of $\partial\AS^{2,1}$ that os the graph of a 1-Lipschitz function $f: \partial\D\to\R$.
\end{definition}
%Def di acausal set in \AS e \A (magari diretto come grafico)
%Lemma di caratt. degli acausal set come grafico
%def di invisible domain 
%def di achronal meridian

% Def di spacelike surface
% Spacelike surfaces sono acausal
%def di domain of dependence e globally hyperbolic spacetime
%prop 4.4.6
%prop 4.5.5
%vedere se aggiungere cose sulla convessità della sezione 4.6

\begin{definition}
    Given a surface $S$ and a Lorentzian manifold $(M,g)$, a $C^1$ immersion $\sigma : S \to M$ is \textit{spacelike} if the pull-back metric $\sigma^* g$ is a Riemannian metric. In this setting $\sigma(S)$ is said to be a \textit{spacelike surface}.
\end{definition}

\begin{definition}
    A \textit{globally hyperbolic spacetime} is a Lorentzian manifold that admits a spacelike surface that meets every casual curve. This surface is callled \textit{Cauchy surface}. 
\end{definition}
Globally hyperbolic spacetime have strong geometric property we summarize in the following theorem:
\begin{theorem}
    Let $M$ be a globally hyperbolic spacetime, then
    \begin{itemize}
        \item Any two Cauchy surfaces are diffeomorphic.
        \item There  exists a submersion $\tau : M \to \R$ whose fibers are Cauchy surfaces.
        \item $M$ is diffeomorphic to $\Sigma \times \R$ where $\Sigma$ is a Cauchy surface.
    \end{itemize}
\end{theorem}

%studiamo MGH al variare del genere della superficie di Cauchy. Ci interessiamo solo di genere >1. Se genere =0 non ci sono globally hyperbolic spacetime.
%Prop 5.1.3, Def 5.1.4 MGH, Cor 5.1.5
%sezione 5.4 e 5.5 su MGH di genere >1
%sezione 6 tutta direi, poi cose sulle foliazioni 