\chapter{Minimal Lagrangian diffeomorphism}

\red{aggiungere intro? aggiungiere sezione su mappe armoniche e minimali}

\section{Harmonic and minimal maps}
In this section we will briefly introduce harmonic and minimal maps, which will prove useful in the proof of the main theorem.\\
For this purpose let $(M,g)$ and $(N,h)$ be Riemannian manifolds. In local coordinates the metric tensors $g$ and $h$ are given by $(g_{\alpha \beta})_{\alpha,\beta = 1,\cdots, m}$ and $(h_{ij})_{i,j= 1,\cdots,n}$. Moreover let $(g^{\alpha \beta}) = (g_{\alpha \beta})^{-1}$ be the inverse metric tensor on $M$ and let $\Gamma^k_{ij}$ be the Christoffel symbols of the Levi Civita connection of $N$.\\
\begin{definition}
    The \textit{energy density} of a map $f:M \to N$ is defined as
    \begin{equation}
        e(f)(x) := \frac{1}{2} g^{\alpha \beta}(x)h_{ij}(f(x)) \frac{\partial f^i (x)}{\partial x^\alpha}  \frac{\partial f^j (x)}{\partial x^\beta} \ .
    \end{equation}
    The \textit{energy} of $f$ is
    \begin{equation}
        E(f) := \int_M e(f)dM
    \end{equation}
    where $dM = \sqrt{det(g)} dx^1 \wedge \cdots \wedge dx^m$ is the volume form on $M$.
\end{definition}
\begin{observation}
    Intrinsically $e(f)$ is the trace (up to the factor $\frac{1}{2}$) of $f^* h$ with respect to $g$, and therefore its value on $x$ does not depend on the choices of local coordinates.
\end{observation}
\begin{definition}
    A map $f:M\to N$ is \textit{harmonic} if it is a critical point of the energy functional.
\end{definition}
As an harmonics map are critical points of $E$, a direct computation of
\[
    \frac{d}{dt}E(f+ t\varphi) |_{t=0} = 0
\]
shows that harmonic maps are the solutions of
\begin{equation}
    \frac{1}{\sqrt{det(g)}} \frac{\partial}{\partial x^\alpha}(\sqrt{det(g)} g^{\alpha\beta} \frac{\partial f^i}{\partial x^\beta}) + g^{\alpha\beta}(x) \Gamma^k_{ij}(f(x)) \frac{\partial f^i}{\partial x^\alpha}  \frac{\partial f^j}{\partial x^\beta} = 0 \ .
\end{equation}

We turn now our attention to the case where $M$ is a compact surface. As a complex manifold $M$ admits complex coordinates $z=x + iy$ which define on the tangent space the vector fields
\[
\frac{\partial}{\partial z} = \frac{1}{2} \Big(\frac{\partial}{\partial x} -i \frac{\partial}{\partial y} \Big) \ , \qquad \frac{\partial}{\partial \overline{z}} = \frac{1}{2} \Big(\frac{\partial}{\partial x} + i \frac{\partial}{\partial y} \Big) ,
\]
and on the cotangent space the 1-forms
\[
dz = dx + idy \ , \qquad d\overline{z} = dx - idy \ .
\]
\begin{definition}
    A Riemannian metric $\langle \cdot , \cdot \rangle$ on a surface $M$ is called $\textit{conformal}$ if in local coordinates it can be written as
    \[
        \rho^2(z) dz \otimes d\overline{z}
    \]
    for a positive, real valued function $\rho(z)$. In particular we have that
    \[
    \begin{split}
        \Big\langle \frac{\partial}{\partial z}, \frac{\partial}{\partial z} \Big\rangle & = 0 = \Big\langle \frac{\partial}{\partial \overline{z}}, \frac{\partial}{\partial \overline{z}} \Big\rangle \ , \\
         \Big\langle \frac{\partial}{\partial z}, \frac{\partial}{\partial \overline{z}} \Big\rangle & = \rho^2(z).
    \end{split}
    \]
\end{definition}
\begin{observation}
    Using real coordinates a conformal metric is written as 
    \[
        \rho^2(z) (dx \otimes dx + dy \otimes dy)
    \]
    and
    \[
    \begin{split}
        \Big\langle \frac{\partial}{\partial x}, \frac{\partial}{\partial x} \Big\rangle & = \rho^2(z) = \Big\langle \frac{\partial}{\partial y}, \frac{\partial}{\partial y} \Big\rangle \ , \\
         \Big\langle \frac{\partial}{\partial x}, \frac{\partial}{\partial y} \Big\rangle & = 0.
    \end{split}
    \]
\end{observation}
\red{------}
\begin{definition}
    Given $M$ a surface and $N$ a Riemannian manifold with a metric $\langle \cdot , \cdot \rangle_N$, a map $f:M \to N$ is called \textit{conformal} if
    \[
    \Big\langle \frac{\partial f}{\partial x}, \frac{\partial f}{\partial x} \Big\rangle_N = \Big\langle \frac{\partial f}{\partial y}, \frac{\partial f}{\partial y} \Big\rangle_N \ , \qquad \Big\langle \frac{\partial f}{\partial x}, \frac{\partial f}{\partial y} \Big\rangle_N = 0 \ .
    \]
    In local coordinates the metric on $N$ can be expressed as $g_{ij}df^i \otimes df^j$ and the conditions of conformality are
    \[
    g_{ij}(f(z)) \frac{\partial f^i}{\partial x} \frac{\partial f^j}{\partial x} = g_{ij}(f(z)) \frac{\partial f^i}{\partial y} \frac{\partial f^j}{\partial y} \ , \qquad g_{ij}(f(z)) \frac{\partial f^i}{\partial x} \frac{\partial f^j}{\partial y} = 0 \ .
    \]
    \red{In complex coordinates these are : vedere se serve e se spostare sta cosa della conformalità -----}
\end{definition}

In complex coordinates Equation \ref{eq:harmonic} are written as
\[
\frac{\partial^2 f^i}{\partial z \partial \overline{z}} + \Gamma^i_{jk} (f(z)) \frac{\partial f^j}{\partial z} \frac{\partial f^k}{\partial \overline{z}} = 0 \qquad \forall i = 1, \cdots, n
\]
When $N$ is a surface with complex coordinates $w = u + iv$ and conformal metric $h(w) = \sigma(w) dw \otimes d\overline(w)$, and $k(z) = f^1(z) + if^2(z)$, Equation \ref{eq:harmonic} are
\begin{equation} \label{eq:harmonic2}
    \frac{\partial k}{ \partial z   \partial \overline{z}} + \frac{1}{\sigma} \frac{\partial \sigma}{ \partial w} \frac{\partial k}{\partial z} \frac{\partial k}{\partial \overline{z}} = 0 \ .
\end{equation}

\begin{theorem}
    Let $M$ be a surface and $N$ a Riemannian manifold with metric $\langle \cdot,\cdot \rangle_N$ or $(g_{ij})_{i,j = 1, \cdots , n}$ in local coordinates. If $f:M \to N$ is harmonic, then 
    \[
        \varphi(z) dz^2 = \Big\langle \frac{\partial f}{\partial z},  \frac{\partial f}{\partial z} \Big\rangle_N dz^z
    \]
    is a holomorphic quadratic differential, i.e $\varphi(z)$ is a holomoprhic funztion. \\
    If $f$ is a diffeomorphism the converse hold.
\end{theorem}
\begin{proof}
    In local coordinates
    \[
        \varphi(z)dz^2 = g_{ij}(f(z)) \frac{\partial f^i}{\partial z}  \frac{\partial f^j}{\partial z} dz^2. 
    \]
    Differentiating in $\overline{z}$ we have
    \[
    \begin{split}
       \frac{ \partial}{\partial \overline{z}} \Big(g_{ij}(f(z)) \frac{\partial f^i}{\partial z}  \frac{\partial f^j}{\partial z} \Big) & = 2 g_{ij} \frac{\partial^2 f^i}{\partial z \partial \overline{z}} \frac{\partial f^j}{\partial z} + \frac{\partial g_{ij}}{\partial x^k} \frac{\partial f^k}{\partial \overline{z}} \frac{\partial f^i}{\partial z} \frac{\partial f^j}{\partial z} \\
       & = 2 g_{ij} \frac{\partial^2 f^i}{\partial z \partial \overline{z}} \frac{\partial f^j}{\partial z} + \Big(\frac{\partial g_{lj}}{\partial x^k} + \frac{\partial g_{lk}}{\partial x^j} + \frac{\partial g_{jk}}{\partial x^l} \Big) \frac{\partial f^k}{\partial \overline{z}} \frac{\partial f^l}{\partial z} \frac{\partial f^j}{\partial z}  \\
       & = 2 g_{ij} \frac{\partial f^j}{\partial z} \Big( \frac{\partial^2 f^i}{\partial z \partial \overline{z}} + \Gamma^i_{kl}  \frac{\partial f^k}{\partial \overline{z}} \frac{\partial f^l}{\partial z} \Big) \ ,
    \end{split}
    \]
    which is $0$ when $f$ is harmonic.
    Let now $N$ be a surface with complex coordinates $w = u + iv$ and conformal metric $h(w) = \sigma(w) dw \otimes d\overline(w)$ and let $k(z) = f^1(z) + if^2(z)$. Now $\varphi(z)$ can be written as $\sigma \frac{\partial k}{ \partial z} \frac{\partial \overline{k}}{\partial z}$ and differentiating in $\overline{z}$ we have
    \[
    \begin{split}
        \frac{ \partial}{\partial \overline{z}}\Big(\sigma(f(z)) \frac{\partial k}{ \partial z}  \frac{\partial \overline{k}}{\partial z} \Big) & = 
        \frac{\partial \sigma}{\partial w} \frac{\partial k}{\partial \overline{z}} \frac{\partial k}{ \partial z} \frac{\partial \overline{k}}{\partial z} +
        \frac{\partial \sigma}{\partial \overline{w}} \frac{\partial \overline{k}}{\partial \overline{z}} \frac{\partial k}{ \partial z} \frac{\partial \overline{k}}{\partial z} +
        \sigma \frac{\partial^2 k}{\partial \overline{z} \partial z} \frac{\partial \overline{k}}{\partial z} +
        \sigma \frac{\partial k}{ \partial z} \frac{\partial^2 \overline{k}}{\partial \overline{z} \partial z} \\
        & = \sigma \frac{\partial \overline{k}}{\partial \overline{z}} \Big( \frac{\partial^2 k}{\partial \overline{z} \partial z} + \frac{1}{\sigma} \frac{\partial \sigma}{\partial w} \frac{\partial k}{\partial \overline{z}} \frac{\partial k}{ \partial z}  \Big) + 
        \sigma \frac{\partial k}{ \partial z} \overline{ \Big(  \frac{\partial^2 k}{\partial \overline{z} \partial z} + \frac{1}{\sigma} \frac{\partial \sigma}{\partial w} \frac{\partial k}{\partial \overline{z}} \frac{\partial k}{ \partial z}  \Big) }  \\
        & = \sigma \frac{\partial \overline{k}}{\partial \overline{z}} P + \sigma \frac{\partial k}{ \partial z} \overline{P} \ ,
    \end{split}
    \]
    where $P$ is the left member of Equation \ref{eq:harmonic2}.\\
    Suppose now that $f$ is a diffeomorphism and that $\varphi$ is holomorphic. We therefore have
    \[
    \frac{ \partial}{\partial \overline{z}}\Big(\sigma(f(z)) \frac{\partial k}{ \partial z}  \frac{\partial \overline{k}}{\partial z} \Big) = 0 \ .
    \]
    If $f$ is not harmonic, $P\neq 0$ and we have
    \[
       \Big| \frac{\partial k}{ \partial z} \Big| = \Big| \frac{\partial \overline{k}}{ \partial z} \Big| = \Big| \frac{\partial k}{ \partial \overline{z}} \Big| \ .
    \]
    This implies that the Jacobian of $f$ is singular which contradics the hypothesis, hence $P=0$ and $f$ is harmonic.
\end{proof}

\section{CMC Surfaces}
A smoothly embedded spacelike surface $S$ has \textit{constant mean curvature} (CMC) if the trace of the shape operator $B$ is constant. A \textit{maximal surface} is a surface $S$ whose mean curvature is constantly equal to zero.
\begin{theorem}\label{thm:CMC_foliation}
    Every MGH Anti-de Sitter manifold with compact Cauchy surface is uniquely foliated by closed CMC surfaces, where the mean curvature $H$ varies in $(-\infty,\infty)$. In particular every MGH Anti-de Sitter manifold with compact Cauchy surface admits a unique maximal surface.
\end{theorem}
The proof of theorem \ref{thm:CMC_foliation} can be found in \cite{barbot2004constant} and revolves around the existence of a pair of barriers, which are two surfaces with everywhere positive/negatime mean curvature, respectively.\\
Given a spacelike immersion $\sigma$, by Lemma \ref{lem:tub metric}
\[
    B_t = (\cos(t) id + \sin(t) B)^{-1}(-\sin(t) id + \cos(t) B).
\]
Let $\lambda(x)$, $\mu(x)$ be the eigenvalues of $B$, by direct computation one find that
\begin{equation} \label{eq:eigenvalues}
    \lambda_t(x) = \frac{\lambda(x)+\tan(t)}{1- \lambda(x)\tan(t)} \ \ \ \ \ \mu_t(x) = \frac{\mu(x)+\tan(t)}{1- \mu(x)\tan(t)} \ .
\end{equation}
In general one can find a relation between CMC and constant Gaussian curvature (CGC) surfaces, namely surfaces where the determinant of the shape operator is constant, see \cite{chen2017constantmeancurvaturefoliation}. Here we show a weaker result which will prove useful in the proof of the main result:
\begin{proposition} \label{prop:cmc to cgc}
    Let $\sigma: S \to\A^{2,1}$ be a maximal surface. Then the normal evolution $\sigma_{t}$ at the time $t= \pi / 4$ is an immersion of constant Gaussian curvature $K = -2$.
\end{proposition}
\begin{proof}
    By Equation \ref{eq:eigenvalues} the Gaussian curvature at $x$ is given by
    \[
        K_t(x) = \frac{\lambda(x)\mu(x) + \lambda(x)\tan(t) + \mu(x)\tan(t) + \tan^2(t)}{1 - \lambda(x)\tan(t)- \mu(x)\tan(t) + \lambda(x)\mu(x)\tan^2(t)}.
    \]
    Since $sigma$ gives a maximal surface, we have $\mu(x) = -\lambda(x)$ and therefore
    \[
        K_t(x) = \frac{\lambda(x)\mu(x) + \tan^2(t)}{1 + \lambda(x)\mu(x)\tan^2(t)}
    \]
    \red{i conti non stanno tornando}
\end{proof}
\red{fare la prop al contrario, per superfici massimali}

\section{Minimal Lagrangian diffeomorphism}

\red{da finire}
\begin{definition}
    Let $h$ and $h'$ be two hyperbolic metrics on the closed surface $S$. A smooth map $f: (S,h) \to (S,h')$ is \textit{minimal Lagrangian} if its graph is a minimal Lagrangian submanifold of $S\times S$ with respect to the product metric $h \oplus h'$ and the symplectic form $\pi_l^*\omega_h - \pi_r^* \omega_{h'}$. 
\end{definition}

\red{|||}
Consider the GMH spacetime $M$ whose holonomy $\rho = (\rho_l, \rho_r)$ induces $h$ and $h'$ as quotient metric in $\H^2 / \rho_l(\pi_1 \Sigma)$ and $\H^2 / \rho_r(\pi_1 \Sigma)$. Let now $\Sigma_0$ the unique maximal surface in $M$, we want to prove that the associated map $f_0$ is a minimal Lagrangian diffeomorphism.\\
First we prove that $f_0$ is a diffeomorphism. By Proposition \ref{prop:cmc to cgc} $\Sigma_0$ is obtained from a surface $\Sigma_1$ of constant Gaussian curvature equal to $-2$. The right and left projection of the Gauss map are always local diffeomorphism on a surface with curvature different from $0$, and by a topological argument the projections of the Gauss map associated to $\Sigma_1$ are diffeomorphism. Now, recalling that spacelike immersions which differ from one another by the normal evolution have the same image, we have that the projections are diffeomorphism for $\Sigma_0$, and therefore $f_0$ is a diffeomorphism. Moreover the Lagrangian condition is equivalent to the fact that $f_0$ is area-preserving, and we proved in Section \ref{sec:gauss map} that this is always verified by the image of the Gauss map.\\
\red{We now want to prove that the Gauss map is harmonic and conformal, which implies that it is minimal.}\\
By proposition \ref{prop:left right pull-back metric}, the pull-back metric of the product Riemannian metric has the expression $2(\I + \III)$. Being $B$ traceless, by the Cayley-Hamilton theorem $B^2 + (\text{det}B)id = 0$. Now, using that $B$ is $\I$-symmetric, we have that $\III = -(\text{det})I$, showing conformality.\\
To prove the harmonicity of the Gauss map it suffices to prove that the left and right projection are harmonic. We check the harmonicity of $\Pi_l$, the proof is the same for $\Pi_r$. To see this we can write the target metric
\[
    \Pi_l^*g_{\H^2}=I((\text{id}-\mathcal{J}B)\cdot,(\text{id}-\mathcal{J}B)\cdot) = \I(\cdot, \cdot) + \III(\cdot, \cdot) - 2\I(\mathcal{J}B \cdot, \cdot),
\]
where we used that $B$ is traceless and therefore $\mathcal{J}B$ is $\I$-symmetric. Moreover $\mathcal{J} B$ is $I$-Codazzi. By \red{lemms} $- 2\I(\mathcal{J}B \cdot, \cdot)$ is the real part of the Hopf differential of $\Pi_l$ which is holomorphic since $\mathcal{J}B$ is $I$-Codazzi. Hence, by \red{lemma harmonic iff HQD} we have that $\Pi_l$ is harmonic.
\red{|||}
