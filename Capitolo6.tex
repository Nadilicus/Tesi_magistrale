\chapter{Minimal Lagrangian diffeomorphism}

\red{Minimal surfaces and particles in 3-manifolds lemma 3.3: ad ogni superficie massimale si può associare un unico GHM spacetime}


\red{aggiungere intro?}
\section{CMC Surfaces}
A smoothly embedded spacelike surface $S$ has \textit{constant mean curvature} (CMC) if the trace of the shape operator $B$ is constant. A \textit{maximal surface} is a surface $S$ whose mean curvature is constantly equal to zero.
\begin{theorem}\label{thm:CMC_foliation}
    Every MGH Anti-de Sitter manifold with compact Cauchy surface is uniquely foliated by closed CMC surfaces, where the mean curvature $H$ varies in $(-\infty,\infty)$. In particular every MGH Anti-de Sitter manifold with compact Cauchy surface admits a unique maximal surface.
\end{theorem}
The proof of theorem \ref{thm:CMC_foliation} can be found in \cite{barbot2004constant} and revolves around the existance of a pair of barriers, which are two surfaces with everywhere positive/negatime mean curvature, respectively.\\
\red{vedere il remark 7.1.3 per bene}\\
Using the normal evolution
\[
    \sigma_t(x) = \text{exp}_{\sigma(x)} (t\nu(x))
\]
one can find a relation between CMC and constanc Gaussian curvature (CGC) surfaces, namely surfaces where the determinant of the shape operator is constant:
\begin{proposition} \label{prop:cmc to cgc}
    Let $\sigma: S \to\A^{2,1}$ be an immersion of constant Gaussian curvature $K>0$. Then the normal evolution $\sigma_{t_K}$ on the convex side of $\sigma$, for time $t_K=\text{arctan}(K^{1/2})$, is an immersion of constant mean curvature $H=K^{-1/2}(K-1)$.
\end{proposition}
\begin{proof}
    By Lemma \ref{lem:tub metric} the first fundamental form $\I_t$ of $\sigma_t$ is given by
    \[
        \I_t(\cdot, \cdot) = \I((\cos(t)id + \sin(t)B)\cdot, (\cos(t)id + \sin(t)B)\cdot).
    \]
    Let us denote by $B_t$ the shape operato of $\I_t$. By Remark \ref{prop:Gaussian curvature} we have
    \[
        K_I = \frac{K_{\I_t}}{\text{det}(\cos(t)id - \sin(t)B_t)} = - \frac{1+\text{det}B_t}{\cos^2(t) + \sin^2(t)\text{det} B_t -\cos(t) \sin(t)\text{tr} B_t}.
    \]
    Hence $K_I$ is constant if and only if $\text{tr} B_t = 2/ \tan(2t)$, that happens for $\tan(t) = 1 / \sqrt{K_\I}$.
\end{proof}
\red{fare la prop al contrario, per superfici massimali}

\section{Minimal Lagrangian diffeomorphism}

\red{da finire}
\begin{definition}
    Let $h$ and $h'$ be two hyperbolic metrics on the closed surface $S$. A smooth map $f: (S,h) \to (S,h')$ is \textit{minimal Lagrangian} if its graph is a minimal Lagrangian submanifold of $S\times S$ with respect to the product metric $h \oplus h'$ and the symplectic form $\pi_l^*\omega_h - \pi_r^* \omega_{h'}$. 
\end{definition}

\red{|||}
Consider the GMH spacetime $M$ whose holonomy $\rho = (\rho_l, \rho_r)$ induces $h$ and $h'$ as quotient metric in $\H^2 / \rho_l(\pi_1 \Sigma)$ and $\H^2 / \rho_r(\pi_1 \Sigma)$. Let now $\Sigma_0$ the unique maximal surface in $M$, we want to prove that the associated map $f_0$ is a minimal Lagrangian diffeomorphism.\\
First we prove that $f_0$ is a diffeomorphism. By Proposition \ref{prop:cmc to cgc} $\Sigma_0$ is obtained from a surface $\Sigma_1$ of constant Gaussian curvature equal to $-2$. The right and left projection of the Gauss map are always local diffeomorphism on a surface with curvature different from $0$, and by a topological argument the projections of the Gauss map associated to $\Sigma_1$ are diffeomorphism. Now, recalling that spacelike immersions which differ from one another by the normal evolution have the same image, we have that the projections are diffeomorphism for $\Sigma_0$, and therefore $f_0$ is a diffeomorphism. Moreover the Lagrangian condition is equivalent to the fact that $f_0$ is area-preserving, and we proved in Section \ref{sec:gauss map} that this is always verified by the image of the Gauss map.\\
\red{We now want to prove that the Gauss map is harmonic and conformal, which implies that it is minimal.}\\
By proposition \ref{prop:left right pull-back metric}, the pull-back metric of the product Riemannian metric has the expression $2(\I + \III)$. Being $B$ traceless, by the Cayley-Hamilton theorem $B^2 + (\text{det}B)id = 0$. Now, using that $B$ is $\I$-symmetric, we have that $\III = -(\text{det})I$, showing conformality.\\
\red{La parte sull'armonicità non mi torna}

%To prove the harmonicity of the Gauss map it suffices to prove that the left and right projection are harmonic. We check the harmonicity of $\Pi_l$, the proof is the same for $\Pi_r$. To see this we can write the target metric
%\[
%    \Pi_l^*g_{\H^2}=I((\text{id}-\mathcal{J}B)\cdot,(\text{id}-\mathcal{J}B)\cdot) = \I(\cdot, \cdot) + \III(\cdot, \cdot) - 2\I(\mathcal{J}B \cdot, \cdot),
%\]
%where we used that $B$ is traceless and therefore $\mathcal{J}B$ is $\I$-symmetric.
\red{da finire}
\red{|||}
%Sep17,  BS10
