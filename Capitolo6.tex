\chapter{Minimal Lagrangian diffeomorphism}

\section{Harmonic and minimal maps}
In this section we will briefly introduce harmonic and minimal maps, which will prove useful in the proof of the main theorem.\\
For this purpose let $(M,g)$ and $(N,h)$ be Riemannian manifolds. In local coordinates the metric tensors $g$ and $h$ are given by $(g_{\alpha \beta})_{\alpha,\beta = 1,\cdots, m}$ and $(h_{ij})_{i,j= 1,\cdots,n}$. Moreover let $(g^{\alpha \beta}) = (g^{-1})_{\alpha \beta}$ be the inverse metric tensor on $M$ and let $\Gamma^k_{ij}$ be the Christoffel symbols of the Levi Civita connection of $(N,h)$.\\
\begin{definition}
    The \textit{energy density} of a smooth map $f:M \to N$ is defined as
    \begin{equation}
        e(f)(x) := \frac{1}{2} g^{\alpha \beta}(x)h_{ij}(f(x)) \frac{\partial f^i (x)}{\partial x^\alpha}  \frac{\partial f^j (x)}{\partial x^\beta} \ .
    \end{equation}
    The \textit{energy} of $f$ is
    \begin{equation}
        E(f) := \int_M e(f)dM
    \end{equation}
    where $dM = \sqrt{det(g)} dx^1 \wedge \cdots \wedge dx^m$ is the volume form on $M$.
\end{definition}
\begin{observation}
    Intrinsically $e(f)$ is the trace (up to the factor $\frac{1}{2}$) of $f^* h$ with respect to $g$, and therefore its value on $x$ does not depend on the choices of local coordinates.
\end{observation}
\begin{definition}
    A map $f:M\to N$ is \textit{harmonic} if it is a critical point of the energy functional.
\end{definition}
As harmonic maps are critical points of $E$, a direct computation of
\[
    \frac{d}{dt}E(f+ t\varphi) |_{t=0} = 0
\]
shows that harmonic maps are the solutions of
\begin{equation} \label{eq:harmonic}
    \frac{1}{\sqrt{det(g)}} \frac{\partial}{\partial x^\alpha}(\sqrt{det(g)} g^{\alpha\beta} \frac{\partial f^i}{\partial x^\beta}) + g^{\alpha\beta}(x) \Gamma^k_{ij}(f(x)) \frac{\partial f^i}{\partial x^\alpha}  \frac{\partial f^j}{\partial x^\beta} = 0 \ .
\end{equation}
We turn now our attention to the case where $M$ is a compact surface. As a complex manifold $M$ admits complex coordinate $z=x + iy$ which define on the complexified tangent space the vector fields
\[
\frac{\partial}{\partial z} = \frac{1}{2} \Big(\frac{\partial}{\partial x} -i \frac{\partial}{\partial y} \Big) \ , \qquad \frac{\partial}{\partial \overline{z}} = \frac{1}{2} \Big(\frac{\partial}{\partial x} + i \frac{\partial}{\partial y} \Big) ,
\]
and on the complexified cotangent space the 1-forms
\[
dz = dx + idy \ , \qquad d\overline{z} = dx - idy \ .
\]
\begin{definition}
    A Riemannian metric $\langle \cdot , \cdot \rangle$ on a surface $M$ is called $\textit{conformal}$ if in local coordinates it can be written as
    \[
        \rho^2(z) dz \wedge d\overline{z}
    \]
    for a positive, real valued function $\rho(z)$. In particular we have that
    \[
    \begin{split}
        \Big\langle \frac{\partial}{\partial z}, \frac{\partial}{\partial z} \Big\rangle & = 0 = \Big\langle \frac{\partial}{\partial \overline{z}}, \frac{\partial}{\partial \overline{z}} \Big\rangle \ , \\
         \Big\langle \frac{\partial}{\partial z}, \frac{\partial}{\partial \overline{z}} \Big\rangle & = 2 \rho^2(z).
    \end{split}
    \]
\end{definition}
\begin{observation}
    Using real coordinates a conformal metric is written as 
    \[
        \rho^2(z) (dx \wedge dx + dy \wedge dy)
    \]
    and
    \[
    \begin{split}
        \Big\langle \frac{\partial}{\partial x}, \frac{\partial}{\partial x} \Big\rangle & = \rho^2(z) = \Big\langle \frac{\partial}{\partial y}, \frac{\partial}{\partial y} \Big\rangle \ , \\
         \Big\langle \frac{\partial}{\partial x}, \frac{\partial}{\partial y} \Big\rangle & = 0.
    \end{split}
    \]
\end{observation}

\noindent In complex coordinates Equation \ref{eq:harmonic} is written as
\[
    \frac{\partial^2 f^i}{\partial z \partial \overline{z}} + \Gamma^i_{jk} (f(z)) \frac{\partial f^j}{\partial z} \frac{\partial f^k}{\partial \overline{z}} = 0 \qquad \forall i = 1, \cdots, n \ .
\]
In particular whether a map is harmonic depends only on the Riemann surface structure of $M$, and not on the choice of a conformal metric.\\
When $N$ is a surface with complex coordinates $w = u + iv$ in a neighbourhood of $f(z)$ and conformal metric $h(w) = \sigma(w) dw \wedge d\overline{w}$, and $f$ is a local deffeomorphism, Equation \ref{eq:harmonic} becomes
\begin{equation} \label{eq:harmonic2}
    \frac{\partial^2 f}{ \partial z   \partial \overline{z}} + \frac{1}{\sigma(f(z))} \frac{\partial \sigma}{ \partial w} \frac{\partial f}{\partial z} \frac{\partial f}{\partial \overline{z}} = 0 \ ,
\end{equation}
where $w(z) = f^1(z) + i f^2(z)$.
\begin{theorem} \label{thm:HQD}
    Let $M$ be a surface and $N$ a Riemannian manifold with metric $\langle \cdot,\cdot \rangle_N$ or $(g_{ij})_{i,j = 1, \cdots , n}$ in local coordinates. If $f:M \to N$ is harmonic, then the \textit{Hopf differential} of $f$
    \[
        \varphi(z) dz^2 = \Big\langle \frac{\partial f}{\partial z},  \frac{\partial f}{\partial z} \Big\rangle_N dz^2
    \]
    is a holomorphic quadratic differential, i.e $\varphi(z)$ is a holomophic function. \\
    If $f$ is a diffeomorphism the converse holds.
\end{theorem}
\begin{proof}
    In local coordinates
    \[
        \varphi(z)dz^2 = g_{ij}(f(z)) \frac{\partial f^i}{\partial z}  \frac{\partial f^j}{\partial z} dz^2. 
    \]
    Differentiating in $\overline{z}$ we have
    \[
    \begin{split}
       \frac{ \partial}{\partial \overline{z}} \Big(g_{ij}(f(z)) \frac{\partial f^i}{\partial z}  \frac{\partial f^j}{\partial z} \Big) & = 2 g_{ij} \frac{\partial^2 f^i}{\partial z \partial \overline{z}} \frac{\partial f^j}{\partial z} + \frac{\partial g_{ij}}{\partial x^k} \frac{\partial f^k}{\partial \overline{z}} \frac{\partial f^i}{\partial z} \frac{\partial f^j}{\partial z} \\
       & = 2 g_{ij} \frac{\partial^2 f^i}{\partial z \partial \overline{z}} \frac{\partial f^j}{\partial z} + \Big(\frac{\partial g_{lj}}{\partial x^k} + \frac{\partial g_{lk}}{\partial x^j} + \frac{\partial g_{jk}}{\partial x^l} \Big) \frac{\partial f^k}{\partial \overline{z}} \frac{\partial f^l}{\partial z} \frac{\partial f^j}{\partial z}  \\
       & = 2 g_{ij} \frac{\partial f^j}{\partial z} \Big( \frac{\partial^2 f^i}{\partial z \partial \overline{z}} + \Gamma^i_{kl}  \frac{\partial f^k}{\partial \overline{z}} \frac{\partial f^l}{\partial z} \Big) \ ,
    \end{split}
    \]
    which is $0$ when $f$ is harmonic.\\
    Let now $N$ be a surface with complex coordinates $w = f^1 + if^2$ in a neighbourhood of $f(z)$ and conformal metric $h(w) = \sigma(w) dw \wedge d\overline{w}$. Now $\varphi(z)$ can be written as $\sigma(f(z)) \frac{\partial f}{ \partial z} \frac{\partial \overline{f}}{\partial z}$ and differentiating in $\overline{z}$ we have
    \[
    \begin{split}
        \frac{ \partial}{\partial \overline{z}}\Big(\sigma(f(z)) \frac{\partial f}{ \partial z}  \frac{\partial \overline{f}}{\partial z} \Big) & = 
        \frac{\partial \sigma}{\partial w} \frac{\partial f}{\partial \overline{z}} \frac{\partial f}{ \partial z} \frac{\partial \overline{f}}{\partial z} +
        \frac{\partial \sigma}{\partial \overline{w}} \frac{\partial \overline{f}}{\partial \overline{z}} \frac{\partial f}{ \partial z} \frac{\partial \overline{f}}{\partial z} +
        \sigma(f(z)) \frac{\partial^2 f}{\partial \overline{z} \partial z} \frac{\partial \overline{f}}{\partial z} +
        \sigma(f(z)) \frac{\partial f}{ \partial z} \frac{\partial^2 \overline{f}}{\partial \overline{z} \partial z} \\
        & = \sigma(f(z)) \frac{\partial \overline{f}}{\partial z} \Big( \frac{\partial^2 f}{\partial \overline{z} \partial z} + \frac{1}{\sigma} \frac{\partial \sigma}{\partial w} \frac{\partial f}{\partial \overline{z}} \frac{\partial f}{ \partial z}  \Big) + 
        \sigma(f(z)) \frac{\partial f}{ \partial z} \overline{ \Big(  \frac{\partial^2 f}{\partial \overline{z} \partial z} + \frac{1}{\sigma} \frac{\partial \sigma}{\partial w} \frac{\partial f}{\partial \overline{z}} \frac{\partial f}{ \partial z}  \Big) }  \\
        & = \sigma(f(z)) \frac{\partial \overline{f}}{\partial z} P + \sigma(f(z)) \frac{\partial f}{ \partial z} \overline{P} \ ,
    \end{split}
    \]
    where $P$ is the left hand side of Equation \ref{eq:harmonic2}.\\
    Suppose now that $f$ is a diffeomorphism and that $\varphi$ is holomorphic. We therefore have
    \[
    \frac{ \partial}{\partial \overline{z}}\Big(\sigma(f(z)) \frac{\partial f}{ \partial z}  \frac{\partial \overline{f}}{\partial z} \Big) = 0 \ .
    \]
    If $f$ is not harmonic, $P\neq 0$.
    Since $|P|=|\overline{P}|$ and $\frac{ \partial}{\partial \overline{z}}\varphi = 0 $, we have
    \[
       \Big| \frac{\partial f}{ \partial z} \Big| = \Big| \frac{\partial \overline{f}}{ \partial z} \Big| = \Big| \frac{\partial f}{ \partial \overline{z}} \Big| \ .
    \]
    This implies that the Jacobian of $f$ is singular which contradics the hypothesis, hence $P=0$ and $f$ is harmonic.
\end{proof}
Before moving on and talk about minimal maps we prove the following result on quadratic differential :
\begin{lemma} \label{lem:HQD}
    Given a Riemannian metric $g$ on a surface $S$ and a $(1,1)$-tensor $A$ which is $g$-symmetric, $A$ is traceless if and only if $g(A \cdot, \cdot)$ is the real part of a quadratic differential for the conformal structure of $g$. Moreover the quadratic differential is holomorphic if and only if A is $g$-Codazzi.
\end{lemma}
\begin{proof}
    Take normal coordinates around a point $x$. In this coordinates $g=\delta_{ij}$ and $A = (A_{ij})_{i,j=1,2}$ where $A_{12}=A_{21}$ since $A$ is $g$-symmetric. Then
    \[
        g(A\cdot , \cdot) = A_{11} (dx^1)^2 + 2 A_{12} dx^1 dx^2 + A_{22} (dx^2)^2 \ .
    \]
    Using complex coordinates $z = x^1 + i x^2$ we have
    \[
        (dx^1)^2 = \frac{dz^2 + 2dzd\overline{z} + d\overline{z}^2}{4} \ , \quad (dx^2)^2 = -\frac{dz^2 - 2dzd\overline{z} + d\overline{z}^2}{4} \ , \quad dx^1dx^2 = i \frac{d\overline{z}^2 - dz^2}{4} \ ,
    \]
    and therefore
    \[
        g(A\cdot , \cdot) = \Big( \frac{A_{11} - A_{22}}{4} - i \frac{A_{12}}{2} \Big)dz^2 + \Big( \frac{A_{11} + A_{22}}{4} \Big)dzd\overline{z} +  \Big( \frac{A_{11} - A_{22}}{4} + i \frac{A_{12}}{2} \Big) d\overline{z}^2 \ .
    \]
    So $g(A\cdot , \cdot)$ is the real part of a quadratic differential if and only if $\frac{A_{11} + A_{22}}{4} = 0$ which happens if and only if $A$ is traceless.\\
    Let now $A$ be traceless, $A_{22}=- A_{11}$ and $g(A \cdot, \cdot)$ writes as
    \[
        g(A\cdot , \cdot) = \Big( \frac{A_{11}}{2} - i \frac{A_{12}}{2} \Big)dz^2 +  \Big( \frac{A_{11}}{2} + i \frac{A_{12}}{2} \Big) d\overline{z}^2 = \varphi dz^2 + \overline{\varphi} d \overline{z}^2 \ .
    \]
    Recall that $d^\nabla A = 0$ means that $\nabla_Y (AX) - \nabla_X (AY) - A(\left[X,Y\right]) = 0$ for all $X,Y \in \Gamma(TS)$, which in normal coordinates is
    \begin{equation}
        Y^i X^j \frac{\partial A_{kj}}{\partial x^i} - X^i Y^j \frac{\partial A_{kj}}{\partial x^i} = 0 \ \ \ \text{for} \ k = 1,2 \ ,
    \end{equation}
    since the Christofell symbols are null. Evaluating $d^\nabla A$ on the pairs $(\frac{\partial}{\partial x^i}, \frac{\partial}{\partial x^j})$ we have that $d^\nabla A = 0$ if and only if
    \[
        \frac{\partial A_{11}}{ \partial x^2} = \frac{\partial A_{12}}{ \partial x^1} \ , \qquad \frac{\partial A_{11}}{ \partial x^1} = - \frac{\partial A_{12}}{ \partial x^2} \ ,
    \]
    which happens if and only if $\varphi$ is holomorphic.
\end{proof}

We now turn our attention to minimal maps.
Let $M$ be a Riemannian manifold, $N$ a Riemannian manifold with metric $\langle \cdot, \cdot \rangle$ and $f:M \to N$ a local embedding. Let $\widetilde{\omega}$ be the volume form on $f(M)$, and $\omega$, $e_1, \cdots , e_m$ a volume form and a frame positively oriented orthonormal frame on $M$. Then
\[
\begin{split}
    \text{Vol}(f(M)) & = \int_{f(M)} \widetilde{\omega} = \int_M f^* \omega  = \int_M | f_* e_1 \wedge \cdots \wedge f_* e_m | \omega = \\
    & = \int_M \langle f_* e_1 \wedge \cdots \wedge f_* e_m ,f_* e_1 \wedge \cdots \wedge f_* e_m \rangle ^{\frac{1}{2}} \omega \ ,
\end{split}
\]
where $\langle v_1 \wedge \cdots \wedge v_m , w_1 \wedge \cdots \wedge w_m \rangle = det(\langle v_i,v_j \rangle)$.\\
Let now $F: M \times (-\epsilon, \epsilon) \to N$ such that $f_t(\cdot) = F(\cdot,t)$ is a local embedding for all $t$, $f_0 = f$ and
\[ supp F:= \overline{ \{ x\in M \ :\ F(x,t) \neq f(x) \ \text{for some} \ t\} }  \]
is a compact subset of $M$. For easier computation suppose that in coordinates $f = id_M$. The variation of volume is then
\[
    \frac{d}{dt} \text{Vol}(f_t(M)) \Big|_{t=0} = \sum_{i=1}^m \int_M \frac{ \langle f_{t*} e_1 \wedge \cdots \wedge \frac{\partial}{\partial t} f_{t*} e_i \wedge \cdots \wedge f_{t*} e_m ,f_{t*} e_1 \wedge \cdots \wedge f_{t*} e_m \rangle }{| f_{t*} e_1 \wedge \cdots \wedge f_{t*} e_m |} \omega
\]
and putting $X =\frac{\partial}{\partial t} f_{t*} |_{t=0} $ we have
\[
\begin{split}
    \frac{d}{dt} \text{Vol}(f_t(M)) \Big|_{t=0} & = \sum_{i=1}^m \int_M \frac{ \langle e_1 \wedge \cdots \wedge \nabla^N_{e_i} X \wedge \cdots \wedge e_m , e_1 \wedge \cdots \wedge e_m \rangle }{| e_1 \wedge \cdots \wedge e_m |} \omega \\
    & = \int_M \langle \nabla^N_{e_i} X, e_i \rangle \ \omega = - \int_M \langle X, \nabla^N_{e_i} e_i \rangle \ \omega \ .
\end{split}
\]
Since the volume in invariant under reparametrization, we can take a parametriztion in which $X^\top =0$ and therefore
\[
   \frac{d}{dt} \text{Vol}(f_t(M)) \Big|_{t=0}  = - \int_M \langle X^\perp, \nabla^N_{e_i} e_i \rangle \ \omega \ . 
\]
\begin{definition}
    A submanifold $M$ of a Riemannian manifold $N$ is called \textit{minimal} if it is a critical point of the volume functional i.e
    \[
        \frac{d}{dt} \text{Vol}(f_t(M)) \Big|_{t=0}  = 0
    \]
    for all local variation of $M$.\\
    Moreover, we say a map $f:M \to N$ is \textit{minimal} is $f(M)$ is a minimal submanifold of $N$.
\end{definition}
Now that the codimension of $M$ can be greater than one, similarly to the case of codimension one, for every choice of a vector field $\nu$ normal to $M$ one has
\[
    B_\nu : TM \to TM
\]
defined by $B_\nu(X) = (\nabla^N_X \nu)^\top$, and the \textit{mean curvature} of $M$ in the direction $\nu$ is
\[
    H_\nu := \frac{1}{m} tr(B_\nu) \ .
\]
\begin{theorem}
    A submanifold $M$ of a Riemannian manifold $N$ is minimal if and only if the mean curvature of $M$ vanishes for all normal directions.
\end{theorem}
\begin{proof}
    We fix an orthonormal basis $\nu_1 , \cdots, \nu_k$ of $T_x M^\perp$ and write $X^\perp = \xi^j \nu_j$.
    Then
    \[
        -\langle X^\perp , \nabla^N_{e_i} e_i \rangle = \xi^j tr(B_{\nu_j}) = m \cdot \xi^j H_{\nu_j} \ ,
    \]
    and $M$ is therefore minimal is and only $m \cdot \xi^j H_{\nu_j}$ vanishes for all choices of $\xi^j$, and the claim follows.
\end{proof}

Let now $f:M \to N$ be an isometric immersion. We want to write the condition for the vanishing of the mean curvature, namely
\[
    (\nabla^N_{e_\alpha} e_\beta)^\perp = 0 \qquad \forall \alpha, \beta \ .
\]
Again we can take normal coordinates on $M$, which, at a point, respect the following
\[
    \nabla^M_{\frac{\partial}{\partial x^\alpha}}  \frac{\partial}{\partial x^\beta} = 0 \ ,
\]
for all $\alpha, \beta = 1, \cdots, m$, and we can take
\[
    e_\alpha = f_*\Big( \frac{\partial}{\partial x^\alpha} \Big) = \frac{\partial f^i}{\partial x^\alpha} \frac{\partial}{\partial f^i} \ ,
\]
where $(f^1, \cdots, f^n)$ are local coordinates near $f(x)$.
Since $f$ is an isometric immersion we have $f_* \nabla^M_X Y = \nabla^{f(M)}_{f_* X} f_* Y$, and
\[
\begin{split}
    (\nabla^N_{e_i} e_i)^\perp & = \nabla^N_{e_i} e_i = \nabla^N_{\frac{\partial f^i}{\partial x^\alpha} \frac{\partial}{\partial f^i}}  \frac{\partial f^j}{\partial x^\alpha} \frac{\partial}{\partial f^j} = \\
    & = \frac{\partial^2 f^j}{(\partial x^\alpha)^2} \frac{\partial}{\partial f^j} + \frac{\partial f^i}{\partial x^\alpha} \frac{\partial f^k}{\partial x^\alpha} \Gamma^j_{ik} \frac{\partial}{\partial f^j} \ ,
\end{split}
\]
where $\Gamma^j_{ik}$ are the Christoffel symbols of $N$. Therefore we conclude that $f(M)$ has vanishing mean curvature if and only if
\begin{equation}\label{eq:minimal}
    \frac{\partial^2 f^j}{(\partial x^\alpha)^2} + \frac{\partial f^i}{\partial x^\alpha} \frac{\partial f^k}{\partial x^\alpha} \Gamma^j_{ik}  = 0 \ \ \ \text{for} \ j=1, \cdots, n \ .
\end{equation}
In arbitrary coordinates Equation \ref{eq:minimal} is written as
\[
    \frac{1}{\sqrt{det(g)}} \frac{\partial}{\partial x^\alpha}(\sqrt{det(g)} g^{\alpha\beta} \frac{\partial f^j}{\partial x^\beta}) + g^{\alpha\beta}(x) \Gamma^j_{ik}(f(x)) \frac{\partial f^i}{\partial x^\alpha}  \frac{\partial f^k}{\partial x^\beta} = 0  \ \ \ \text{for} \ j=1, \cdots, n \ ,
\]
where $(g_{\alpha \beta})_{\alpha,\beta = 1,\cdots,m}$ is the metric tensor on $M$.
We can therefore conclude that an isometric immersion is minimal if and only if it is harmonic.\\
Now let $M$ be a surface with metric $g$ and $N$ a Riemannian manifold with metric $h$. We say that a map $f:M \to N$ is \textit{conformal} if $f^* h = \lambda g$ where $\lambda$ is some positive function, namely the pull-back of the metric $h$ is \textit{conformal} to $g$. We already observed that the harmonicity of $f:M \to N$ when $M$ is a surface does not depend on the choice of conformal metric, we can therefore observe that an immersion $f:M \to N$ that is conformal and harmonic, is minimal. It suffices to substitute the metric $g$ on $M$ with the conformal metric $f^* h$, making $f$ an isometric immersion, and now the fact that $f$ is harmonic is now equivalent to it being minimal.\\
To summarize we state the following
\begin{theorem}\label{thm:minimal}
    Let $M$ be a surface with metric $g$ and $N$ a Riemmanian manifold. If $f: M \to N$ is a conformal, harmonic immersion then it is minimal.
\end{theorem}

\section{CMC Surfaces}
We now turn back out attention to smoothly embedded spacelike surfaces in globally hyperbolic AdS spacetime. A surface $S$ has \textit{constant mean curvature} (CMC) if the trace of the shape operator $B$ is constant. A \textit{maximal surface} is a surface $S$ whose mean curvature is constantly equal to zero.
\begin{theorem}\label{thm:CMC_foliation}
    Every MGH Anti-de Sitter manifold with compact Cauchy surface admits a maximal surface.
\end{theorem}
The proof of Theorem \ref{thm:CMC_foliation} revolves around the existence of a pair of barriers which implies existence of a maximal surface. A pair of barriers is a pair of surfaces $\Sigma^+$, $\Sigma^-$ with everywhere positive/negative mean curvature, respectively, and such that $\Sigma^+$ is contained in the future of $\Sigma^-$.
To do this one can take convex hull $C$ of a proper achronal meridian $\Lambda$ that defines $M$. This definition makes sense in affine chart, and in this setting it is well defined since $C \subset \Omega(\Lambda)$ which is contained in a Dirichlet domain. Now $\partial C$ is the union of $\Lambda$ and two surfaces $\partial_+ C$, $\partial_- C$ which are convex and concave respectively. Here a set is convex (resp. concave) if it is contained in the future (resp. past) of all its support planes. Convex (resp. concave) surfaces have non-positive (resp. non-negative) mean curvature. The quotient of $\partial_+ C$, $\partial_- C$ do not work as barriers since they are not smooth, but close to them one can find a pair of barriers.
A detailed proof can be found in \cite{barbot2004constant}.\\
Given a spacelike immersion $\sigma$, by Lemma \ref{lem:tub metric}
\[
    B_t = (\cos(t) id + \sin(t) B)^{-1}(-\sin(t) id + \cos(t) B).
\]
Let $\lambda(x)$, $\mu(x)$ be the eigenvalues of $B$. By direct computation one find that
\begin{equation} \label{eq:eigenvalues}
    \lambda_t(x) = \frac{\lambda(x)+\tan(t)}{1- \lambda(x)\tan(t)} \qquad \mu_t(x) = \frac{\mu(x)+\tan(t)}{1- \mu(x)\tan(t)} \ .
\end{equation}
In general one can find a relation between CMC surfaces and constant Gaussian curvature (CGC) surfaces, namely surfaces where $K = -1 - detB$ is constant, see \cite{chen2017constantmeancurvaturefoliation}. Here we show a weaker result which will prove useful in the proof of the main result:
\begin{proposition} \label{prop:cmc to cgc}
    Let $\sigma: S \to\A^{2,1}$ be a maximal surface. Then the normal evolution $\sigma_{t}$ at the time $t= -\pi / 4$ is an immersion of constant Gaussian curvature $K = -2$.
\end{proposition}
\begin{proof}
    Since $\sigma$ determines a maximal surface $\mu(x) = - \lambda(x)$. By Equation \ref{eq:eigenvalues} the eigenvalues of $B_t$ at time $t=-\pi / 4$ are
    \[
        \lambda_{-\frac{\pi}{4}}(x) = \frac{\lambda(x) -1}{1 +\lambda(x)} \qquad \mu_{-\frac{\pi}{4}}(x) = \frac{\mu(x)-1}{1 + \mu(x)} = \frac{\lambda(x) + 1}{-1 + \lambda(x)} \ .
    \]
    The Gaussian curvature at $x$ is therefore given by
    \[
        K_{-\frac{\pi}{4}}(x) = - 1 - \frac{ \lambda^2(x) -1}{ \lambda^2(x) -1 } = -1 -1 = -2 \ .
    \]
\end{proof}

\section{Minimal Lagrangian diffeomorphisms}

\begin{definition}
    Let $h$ and $h'$ be two hyperbolic metrics on the closed surface $S$. A smooth map $f: (S,h) \to (S,h')$ is \textit{minimal Lagrangian} if its graph is a minimal Lagrangian submanifold of $S\times S$ with respect to the product metric $h \oplus h'$ and the symplectic form $\pi_l^*\omega_h - \pi_r^* \omega_{h'}$. 
\end{definition}

\begin{theorem}\label{thm:existence}
    Given a closed surface $\Sigma$ and two hyperbolic metrics $h$, $h'$ on it, there exists a minimal Lagrangian diffeomorphism $f_0 : (\Sigma,h) \to (\Sigma,h')$ isotopic to the identity.
\end{theorem}
\begin{proof}
    Consider the MGH spacetime $M$, with compact Cauchy surface $\Sigma$, whose holonomy $\rho = (\rho_l, \rho_r)$ induces $h$ and $h'$ as quotient metric in $\H^2 / \rho_l(\pi_1 \Sigma)$ and $\H^2 / \rho_r(\pi_1 \Sigma)$. Let now $\Sigma_0$ be a maximal surface in $M$, we want to prove that the associated map $f_0$ is a minimal Lagrangian diffeomorphism.\\
    First we prove that $f_0$ is a diffeomorphism. By Proposition \ref{prop:cmc to cgc}, $\Sigma_0$ is obtained from a surface $\Sigma_1$ of constant Gaussian curvature equal to $-2$ by normal evolution. The right and left projection of the Gauss map are always local diffeomorphism on a surface with curvature everywhere different from $0$, and by a topological argument the projections of the Gauss map associated to $\Sigma_1$ are diffeomorphisms. Now, recalling that spacelike immersions which differ from one another by the normal evolution have the same image, we have that the projections are diffeomorphism for $\Sigma_0$, and therefore $f_0$ is a diffeomorphism. Moreover the Lagrangian condition is equivalent to the fact that $f_0$ is area-preserving, and we proved in Section \ref{sec:gauss map} that this is always verified by the image of the Gauss map.\\
    We now want to prove that the Gauss map is harmonic and conformal, which, by Theorem \ref{thm:minimal} implies that it is minimal.\\
    The pull-back metric via the Gauss map is the sum of $\Pi_l^*g_{\H^2}$ and $\Pi_r^*g_{\H^2}$, which, by proposition \ref{prop:left right pull-back metric}, is given by
    \[
       \Pi_l^*g_{\H^2} + \Pi_r^*g_{\H^2} = 2I + 2I(\mathcal{J}B,\mathcal{J}B) = 2I + 2I(B,B)= 2(I+ \III) \ .
    \]
    Being $B$ traceless, by the Cayley-Hamilton theorem $B^2 + (\text{det}B)id = 0$. Now, using that $B$ is $\I$-symmetric, we have that $\III = -(\text{det})I$, showing conformality.\\
    To prove the harmonicity of the Gauss map it suffices to prove that the left and right projection are harmonic. We check the harmonicity of $\Pi_l$, since the proof is the same for $\Pi_r$. To see this we can write the target metric
    \[
        \Pi_l^*g_{\H^2}=I((\text{id}-\mathcal{J}B)\cdot,(\text{id}-\mathcal{J}B)\cdot) = \I(\cdot, \cdot) + \III(\cdot, \cdot) - 2\I(\mathcal{J}B \cdot, \cdot),
    \]
    where we used that $B$ is traceless and therefore $\mathcal{J}B$ is $\I$-symmetric. Moreover $\mathcal{J} B$ is $I$-Codazzi and by Lemma \ref{lem:HQD} $- 2\I(\mathcal{J}B \cdot, \cdot)$ is the real part of the Hopf differential of $\Pi_l$ which is holomorphic since $\mathcal{J}B$ is $I$-Codazzi. Hence, by Theorem \ref{thm:HQD} we have that $\Pi_l$ is harmonic.
\end{proof}

For the uniqueness of the minimal Lagrangian diffeomorphism we need the following:
\red{mettere il lemma}

We are now ready to prove the uniqueness
\begin{theorem}
    Given a surface $\Sigma$ and two hyperbolic metrics $h, h'$ on it, the minimal Lagrangian diffeomorphism $\Phi : (\Sigma, h) \to (\Sigma, h')$ is unique.
\end{theorem}
\begin{proof}
    Let $b$ be the unique $(1,1)$-tensor associated to $\Phi$ as in \red{Lemma da mettere}. We now want to construct a pair $(\I,\II)$ that define a maximal immersion such that the associated map $f_0$ respect $f_0^* h' = h(b \cdot, b \cdot)$. Since by \red{unicità sup max} and we know by Theorem \ref{thm:existence} that $f_0$ is a minimal Lagrangian deffeomorphism, we will have that $f_0 = \Phi$.\\
    Let $\I$ be the metric on $\Sigma$ defined by
    \[
        4 \I = h((id + b)\cdot , (id + b)\cdot ) \ ,
    \]
    let $\mathcal{J}$ be the complex structure defined on $\Sigma$ by $\I$ and $B$ the $(1,1)$-tensor defined by
    \[
        \mathcal{J} B = (id + b)^{-1} (b-id) \ .
    \]
    Here $\I$ and $B$ are well-defined at all points since $b$ has positive eigenvalues. Since $\text{det}(b)=1$, a direct computation in a basis where $b$ is diagonal shows that $\text{tr}(\mathcal{J} B) = 0 $, hence $B$ is $\I$-symmetric. Hence we can set $\II = \I (B \cdot, \cdot)$. We can also write $b$ as
    \[
        b = (id+\mathcal{J}B)(id - \mathcal{J}B)^{-1} \ .
    \]  
    We now have to prove that $\I$ and $B$ satisfy the Gauss-Codazzi equations, and that $B$ is traceless.
    $\mathcal{J} B$ is $\I$-symmetric, indeed for $v,w$ tangent vectors we have
    \[
    \begin{split}
        4I(\mathcal{J}B v,w) & = h((id + b) (id + b)^{-1} (b-id) v , (id + b)w) \\
        & = h((b-id) v , (id + b)w) = h((id + b) v , (b-id)w) \\
        & = 4 I (v, \mathcal{J}B w) \ ,
    \end{split}
    \]
    and it follows that $B$ is traceless.\\
    By Proposition \ref{prop:Gaussian curvature} the Levi-Civita connection $\nabla$ of $\I$ is
    \[
        \nabla^I_Y X = (id + b)^{-1} \nabla^h_Y ((id+b)X) \ .
    \]
    Since $b$ is $h$-Codazzi, $(b-id)$ is $h$-Codazzi, and therefore
    \[
    \begin{split}
        (d^{\nabla^I} \mathcal{J}B)(X,Y) & = (id + b)^{-1} (\nabla^h_Y ((id+b)\mathcal{J}B X) - \nabla^h_X ((id+b)\mathcal{J}B Y) ) - \mathcal{J}B \left[ X,Y \right] \\
        & = (id + b)^{-1} (\nabla^h_Y ((b-id) X) -  \nabla^h_X ((b-id) Y) - (b-id) \left[ X,Y \right] ) \\
        & = (id + b)^{-1}(d^{\nabla^h}(b-id))(X,Y) = 0 \ .
    \end{split}
    \]
    Now, since $\mathcal{J}\nabla^I_Y(BX) = (\nabla^I_Y \mathcal{J})(BX)$, we have that $d^{\nabla^I} (\mathcal{J}B) = \mathcal{J} d^{\nabla^I} B$, and therefore $B$ is $h$-Codazzi.\\
    A simple computation shows that 
    \begin{equation}\label{eq:roba}
        (id+ \mathcal{J}B) = 2(id +b)^{-1} b  \ , \qquad (id- \mathcal{J}B) = 2(id +b)^{-1} \ ,
    \end{equation}
     hence, by Theorem \ref{prop:Gaussian curvature} the curvature of $I$ is given by
    \[
        K_I = \frac{-1}{\det((id+b)/2)} = -\det(id - \mathcal{J}B) = -1 - \det(B) \ .
    \]
    We have shown that the pair $(\I,\II)$ defines a maximal immersion.
    Let now $f_0$ be the minimal Lagrangian diffeomorphism associated to the pair $(I,\II)$, the $(1,1)$ tensor associated to $f_0$ in \red{Lemma da mettere} is clearly $b_{f_0} = (id+\mathcal{J}B)(id-\mathcal{J}B)^{-1} = b$, hence $f_0 = \Phi$. The uniqueness of the maximal surface ensures that any other minimal Lagrangian diffeomorphism $\Phi'$ is equal, via the same construction, to $f_0=\Phi$, hence the uniqueness.
\end{proof}
